\documentclass[11pt]{article}
\usepackage[margin=1in]{geometry}
\usepackage{amsmath,amssymb,amsthm,mathtools}
\usepackage{enumitem}
\usepackage{hyperref}

\newcommand{\E}{\mathcal{E}}
\newcommand{\Var}{\mathrm{Var}}
\newcommand{\1}{\mathbf{1}}
\newcommand{\Sg}{\mathfrak{S}}
\newcommand{\rw}{\mathrm{RW}}
\newcommand{\ip}{\mathrm{IP}}
\newcommand{\ex}{\mathrm{EX}}
\newcommand{\gap}{\lambda_1}

\title{Aldous' Spectral Gap Conjecture (CLR): Core Ideas With and Without Holes}
\author{}
\date{}

\begin{document}
\maketitle
\vspace{-1.0em}

\paragraph{Goal.}
Give a self-contained ``dictionary'' for the statement and proof spine of Aldous' spectral gap conjecture and its extension to exclusion (``holes'').

\section*{1.\ Reversible generators, Dirichlet forms, and barriers}

Let $(\Omega,\pi)$ be a finite probability space. A continuous-time reversible Markov chain has generator
\[
(L f)(\omega)=\sum_{\omega'\neq \omega} q(\omega,\omega')\bigl(f(\omega')-f(\omega)\bigr),
\qquad
\pi(\omega)q(\omega,\omega')=\pi(\omega')q(\omega',\omega).
\]
The Dirichlet form and gap (Poincar\'e constant) are
\[
\E_L(f)=\frac12\sum_{\omega,\omega'}\pi(\omega)q(\omega,\omega')\bigl(f(\omega')-f(\omega)\bigr)^2,
\qquad
\gap(L)=\inf_{f:\Var_\pi(f)\neq 0}\frac{\E_L(f)}{\Var_\pi(f)}.
\]
\paragraph{Barrier upper bound.}
For any $A\subseteq \Omega$, taking $f=\1_A$ gives
\[
\gap(L)\le \frac{\sum_{\omega\in A,\ \omega'\notin A}\pi(\omega)q(\omega,\omega')}
{\pi(A)\bigl(1-\pi(A)\bigr)}.
\]
Thus a set $A$ with moderate mass but tiny boundary flow produces a small gap (a \emph{Dirichlet barrier}).

\paragraph{Quadratic-form comparison.}
If two reversible generators $L_1,L_2$ share the same stationary $\pi$ and satisfy
$\E_{L_1}(f)\ge \E_{L_2}(f)$ for all $f$, then $\gap(L_1)\ge \gap(L_2)$.

\section*{2.\ Three chains on a weighted graph}

Fix a finite vertex set $V$, $|V|=n$, and symmetric edge rates $c_{xy}=c_{yx}\ge 0$ (with $c_{xy}=0$ allowed).

\subsection*{2.1.\ Random walk (RW)}
State space $\Omega_{\rw}=V$ with generator
\[
(L_{\rw} f)(x)=\sum_{y\neq x} c_{xy}\,(f(y)-f(x)).
\]
Write $\gap^{\rw}(G):=\gap(L_{\rw})$.

\subsection*{2.2.\ Interchange process (IP) (no holes)}
State space $\Omega_{\ip}=\Sg_n$ (permutations / bijective labelings of $V$). Let $\tau_{xy}$ swap labels at $x,y$. Then
\[
(L_{\ip} F)(\eta)=\sum_{x<y} c_{xy}\bigl(F(\eta\tau_{xy})-F(\eta)\bigr),
\]
reversible w.r.t.\ the uniform measure $\nu$ on $\Sg_n$. Write $\gap^{\ip}(G):=\gap(L_{\ip})$.

\paragraph{Easy inequality (factor to RW).}
Track one distinguished label under IP; it performs RW with rates $c_{xy}$. Hence
\[
\gap^{\ip}(G)\le \gap^{\rw}(G).
\]

\subsection*{2.3.\ Symmetric exclusion $\ex_k$ (with holes)}
Fix $k\in\{1,\dots,n-1\}$. State space
\[
\Omega_{\ex,k}=\{\zeta\subseteq V:\ |\zeta|=k\},
\]
interpreting $\zeta$ as the occupied sites (particles) and $V\setminus \zeta$ as holes. Generator:
\[
(L_{\ex,k} f)(\zeta)=\sum_{x\in \zeta}\sum_{y\notin \zeta} c_{xy}\Bigl(f(\zeta\cup\{y\}\setminus\{x\})-f(\zeta)\Bigr),
\]
reversible w.r.t.\ uniform on $\Omega_{\ex,k}$. For $k=1$, this is exactly RW. Exclusion is a factor of IP:
select $k$ labels as ``particles'' and forget their identities, keeping only the occupied set.

\section*{3.\ Aldous' conjecture and what it means}

\paragraph{Aldous' spectral gap conjecture (proved by CLR).}
For every finite weighted graph $G$,
\[
\boxed{\ \gap^{\ip}(G)=\gap^{\rw}(G)\ }.
\]
Equivalently: the slowest nontrivial mode of interchange is already visible in the one-particle projection.

\paragraph{With holes.}
A parallel conclusion holds for symmetric exclusion: the gap of $\ex_k$ equals the RW gap (for each $k$), so adding many particles subject to exclusion introduces no slower relaxation mode than the one-particle mode.
(Precise statements and proofs are treated alongside interchange in \cite{CLR}.)

\section*{4.\ Proof spine (CLR): elimination + octopus + induction}

The heart of CLR is a Dirichlet-form domination that simulates electrical reduction.

\subsection*{4.1.\ Vertex elimination (star--mesh / Kron reduction) for RW}
Fix $x\in V$ and write $V_x=V\setminus\{x\}$. Define a reduced weighted graph $G_x$ on $V_x$ with
\[
\tilde c_{yz}=c_{yz}+c^{*,x}_{yz},
\qquad
c^{*,x}_{yz}=\frac{c_{xy}c_{xz}}{\sum_{w\in V_x}c_{xw}},
\qquad y\neq z\in V_x.
\]
A key monotonicity for RW is
\[
\gap^{\rw}(G_x)\ \ge\ \gap^{\rw}(G),
\]
i.e.\ eliminating a vertex and replacing its star by a clique can only increase the RW gap.

\subsection*{4.2.\ The octopus inequality (Dirichlet comparison for IP)}
For any $F:\Sg_n\to \mathbb{R}$ and any fixed $x\in V$,
\[
\boxed{\ \sum_{y\in V_x} c_{xy}\,\nu\!\left[\bigl(F(\eta\tau_{xy})-F(\eta)\bigr)^2\right]
\ \ge\
\sum_{\substack{y<z\\ y,z\in V_x}} c^{*,x}_{yz}\,\nu\!\left[\bigl(F(\eta\tau_{yz})-F(\eta)\bigr)^2\right]\ }.
\]
Interpretation: the collection of star transpositions touching $x$ already contains (in quadratic-form strength)
the induced clique transpositions on $V_x$, with \emph{exactly} the Kron reduction coefficients.

\subsection*{4.3.\ Induction closes the theorem}
Sketch the logic (details in \cite{CLR}):
\begin{enumerate}[leftmargin=1.4em,itemsep=0.2em]
\item Apply the octopus inequality to compare the IP Dirichlet form on $G$ to the IP Dirichlet form on the reduced graph $G_x$.
\item By induction on $n$, assume $\gap^{\ip}(G_x)=\gap^{\rw}(G_x)$.
\item Use RW elimination monotonicity $\gap^{\rw}(G_x)\ge \gap^{\rw}(G)$.
\item Conclude $\gap^{\ip}(G)\ge \gap^{\rw}(G)$; combined with the easy inequality yields equality.
\end{enumerate}

\paragraph{Takeaway.}
RW has a literal star--mesh reduction. CLR proves the octopus inequality to make an \emph{analogous reduction true at the level of IP Dirichlet forms}, enabling an induction on the vertex count.

\section*{5.\ Minimal mental model: ``without holes'' vs.\ ``with holes''}

\begin{itemize}[leftmargin=1.3em,itemsep=0.2em]
\item \textbf{No holes (IP):} state space is huge ($n!$), stationary measure is uniform, updates are edge transpositions.
Aldous says the first nonzero eigenvalue matches RW.
\item \textbf{With holes (exclusion):} state space is $\binom{n}{k}$, uniform stationary measure, updates move particles across edges.
The gap still matches RW: exclusion introduces constraints but not a smaller spectral gap than the one-particle mode.
\item \textbf{Why this is nontrivial:} a priori, multi-label / multi-particle systems could have new bottlenecks invisible to a single particle. CLR shows they do not: the one-particle sector already controls the gap.
\end{itemize}


\pagebreak 


% =========================
% Part I (Aldous / CLR): What to present + how the crucial inequality is proved
% Drop-in raw LaTeX sections for meeting plan
% =========================

\section{Part I: Aldous/CLR as the Spectral-Geometry Foundation}

\subsection{Objects: Two Markov chains on a weighted graph}

Fix a finite vertex set $V$ with $|V|=n$ and symmetric edge rates $c_{xy}=c_{yx}\ge 0$ (allow $c_{xy}=0$).
We will compare the spectral gaps of two reversible continuous-time chains built from the same edge rates.

\paragraph{Random walk (RW).}
State space $\Omega_{\rw}=V$ with generator
\[
(L_{\rw} f)(x)=\sum_{y\neq x} c_{xy}\,\bigl(f(y)-f(x)\bigr).
\]
Write $\lambda^{\rw}(G):=\lambda_1(L_{\rw})$.

\paragraph{Interchange process (IP).}
State space $\Omega_{\ip}=\mathfrak{S}_n$ (permutations/labelings of the vertices).
Let $\tau_{xy}$ denote the transposition swapping the labels at vertices $x,y$.
Then
\[
(L_{\ip} F)(\eta)=\sum_{x<y} c_{xy}\,\bigl(F(\eta\tau_{xy})-F(\eta)\bigr),
\]
reversible with respect to the uniform measure $\nu$ on $\mathfrak{S}_n$.
Write $\lambda^{\ip}(G):=\lambda_1(L_{\ip})$.

\subsection{The proof interface: Dirichlet forms and spectral gaps}

For a finite reversible chain $(\Omega,\pi,L)$, the Dirichlet form and gap are
\[
\mathcal{E}_L(f)=\frac12\sum_{\omega,\omega'}\pi(\omega)q(\omega,\omega')\bigl(f(\omega')-f(\omega)\bigr)^2,
\qquad
\lambda_1(L)=\inf_{f:\ \mathrm{Var}_\pi(f)\neq 0}\frac{\mathcal{E}_L(f)}{\mathrm{Var}_\pi(f)}.
\]
For IP we can write
\[
\mathcal{E}_{\ip}(F)=\frac12\sum_{a<b} c_{ab}\,\nu\!\left[\bigl(\nabla_{ab}F\bigr)^2\right],
\qquad
\nabla_{ab}F(\eta):=F(\eta\tau_{ab})-F(\eta).
\]

\paragraph{Easy direction (factor map).}
Tracking a single label under IP yields RW as a factor. Consequently,
\[
\lambda^{\ip}(G)\le \lambda^{\rw}(G).
\]
Thus the substantive task is the reverse inequality $\lambda^{\ip}(G)\ge \lambda^{\rw}(G)$.

\subsection{Aldous' conjecture (CLR theorem): statement and takeaway}

\paragraph{Theorem (Caputo--Liggett--Richthammer; Aldous' conjecture).}
For every finite weighted graph $G$,
\[
\boxed{\ \lambda^{\ip}(G)=\lambda^{\rw}(G)\ }.
\]
\emph{Takeaway for expanders:} if $G$ has a constant RW gap (e.g.\ a bounded-degree Ramanujan expander), then the many-body IP also has constant gap. In particular, the interchange ``move graph'' does not introduce new bottlenecks beyond those already visible to a single particle.

\subsection{The induction template: vertex elimination and the need for an IP analogue}

Fix a vertex $x\in V$, and set $V_x:=V\setminus\{x\}$ and
\[
c_x:=\sum_{w\in V_x} c_{xw}.
\]
Define the Kron-reduction (star--mesh) weights on $V_x$ by
\[
c^{*,x}_{yz}:=\frac{c_{xy}c_{xz}}{c_x},
\qquad y,z\in V_x,\ y\neq z,
\]
and the reduced graph $G_x$ on $V_x$ with edge weights
\[
\tilde c_{yz}:=c_{yz}+c^{*,x}_{yz}.
\]

\paragraph{RW elimination monotonicity.}
A classical fact from electrical network theory is:
\[
\lambda^{\rw}(G_x)\ \ge\ \lambda^{\rw}(G).
\]
Thus, if we can compare the \emph{IP} Dirichlet form on $G$ to the \emph{IP} Dirichlet form on $G_x$ with the \emph{same coefficients} $c^{*,x}_{yz}$, then an induction on $n$ becomes possible.

\subsection{The crucial inequality: the octopus inequality}

\paragraph{Octopus inequality (Dirichlet-form domination).}
For every function $F:\mathfrak{S}_n\to\mathbb{R}$ and every vertex $x\in V$,
\[
\boxed{\ \sum_{y\in V_x} c_{xy}\,\nu\!\left[\bigl(\nabla_{xy}F\bigr)^2\right]
\ \ge\
\sum_{\substack{y<z\\ y,z\in V_x}} c^{*,x}_{yz}\,\nu\!\left[\bigl(\nabla_{yz}F\bigr)^2\right]\ }.
\]
\emph{Interpretation:} the collection of ``tentacle'' swaps touching $x$ controls the effective clique swaps among $V_x$ with exactly the Kron-reduction weights. This is the IP analogue of star--mesh reduction.

\paragraph{3-vertex sanity check.}
If $V=\{x,y,z\}$, then $c^{*,x}_{yz}=\frac{c_{xy}c_{xz}}{c_{xy}+c_{xz}}$ and octopus becomes
\[
c_{xy}\,\nu[(\nabla_{xy}F)^2] + c_{xz}\,\nu[(\nabla_{xz}F)^2]
\ \ge\
\frac{c_{xy}c_{xz}}{c_{xy}+c_{xz}}\ \nu[(\nabla_{yz}F)^2],
\]
matching the effective-conductance intuition.

\subsection{How octopus is proved (what to say on the board)}

The proof is not a ``canonical paths'' argument; it is an \emph{operator positivity} statement.
Define the difference operator
\[
\mathcal{O}_x
\ :=\
\sum_{y\in V_x} c_{xy}\,(I-\tau_{xy})
\ -\
\sum_{\substack{y<z\\ y,z\in V_x}} c^{*,x}_{yz}\,(I-\tau_{yz}).
\]
Then octopus is equivalent to
\[
\langle F,\mathcal{O}_x F\rangle_{L^2(\nu)}\ \ge\ 0\qquad \forall F,
\]
i.e.\ $\mathcal{O}_x\succeq 0$ in the $L^2(\nu)$ quadratic-form sense.
CLR (and later expositions) establish this by rewriting $\mathcal{O}_x$ as a sum of positive semidefinite pieces (a Gram/sum-of-squares decomposition), which makes the inequality uniform in $F$.

\subsection{Closing Aldous via induction (the 4-line proof spine)}

\begin{enumerate}[leftmargin=1.6em,itemsep=0.25em]
\item Use octopus to compare IP Dirichlet contributions supported on the star at $x$ to the induced clique on $V_x$ with weights $c^{*,x}_{yz}$.
\item Apply the induction hypothesis on the reduced graph $G_x$:
$\lambda^{\ip}(G_x)=\lambda^{\rw}(G_x)$.
\item Use RW elimination monotonicity:
$\lambda^{\rw}(G_x)\ge \lambda^{\rw}(G)$.
\item Deduce $\lambda^{\ip}(G)\ge \lambda^{\rw}(G)$; combine with the easy inequality to conclude equality.
\end{enumerate}

\subsection{``With holes'': exclusion as a factor (optional 1-slide interlude)}

Fix $k\in\{1,\dots,n-1\}$. The symmetric exclusion process (holes model) has state space
\[
\Omega_{\ex,k}=\{\zeta\subseteq V:\ |\zeta|=k\},
\]
and generator
\[
(L_{\ex,k} f)(\zeta)=\sum_{x\in \zeta}\sum_{y\notin \zeta} c_{xy}\Bigl(f(\zeta\cup\{y\}\setminus\{x\})-f(\zeta)\Bigr),
\]
reversible with respect to the uniform measure on $\Omega_{\ex,k}$.
It is a factor of IP: select $k$ distinguished labels as particles and forget their identities, retaining only the occupied set.
The philosophical consequence consistent with Aldous is that the slowest mode is already present in the one-particle sector (RW), i.e.\ multi-particle constraints do not introduce a smaller spectral gap than RW.

% =========================
% Segue section: explicitly hand off from Aldous/CLR to your new ideas (move graph + potential well)
% =========================

\section{Segue: What Aldous Gives Us, and What Must Be New in the QMC Move-Graph Setting}

\subsection{What Aldous/CLR gives (template)}

The CLR theorem gives a \emph{sharp} mechanism for certifying that a huge configuration-space dynamics has the same spectral gap as the underlying physical graph:
\begin{itemize}[leftmargin=1.6em,itemsep=0.25em]
\item Identify a large state space with \emph{local} generators built from edge swaps (IP / exclusion).
\item Prove an \emph{elimination-style Dirichlet domination} (octopus) mirroring Kron reduction for RW.
\item Induct on the number of vertices to show the first nonzero eigenvalue is inherited from the 1-particle sector.
\end{itemize}
On an expander, this yields a constant spectral gap in the corresponding ``move graph'' dynamics.

\subsection{Why our problem is different (where octopus does \emph{not} directly apply)}

In the QMC / VB / perfect-matching move graph:
\begin{itemize}[leftmargin=1.6em,itemsep=0.25em]
\item The state space is not $\mathfrak{S}_n$ but a constrained combinatorial manifold (e.g.\ perfect matchings / VB coverings).
\item Allowed moves are not arbitrary transpositions but local $4$-cycle flips (``dimer swaps'').
\item Most critically, the effective weight is typically \emph{not uniform}: diagonal energies create a potential well and (in loop/SSE formulations) nonlocal topology can introduce global factors.
\end{itemize}
Therefore, the correct object becomes a \emph{Dirichlet eigenvalue / barrier} statement for a \emph{weighted} chain on the move graph: we must control how the potential (diagonal term) interacts with kinetic expansion (off-diagonal connectivity).

\subsection{The handoff (one sentence)}

\emph{Aldous/CLR provides the spectral-geometry template: local-to-global control via Dirichlet-form inequalities.
Our new contribution is to build the analogue for the matching move graph in the presence of a potential well, yielding a Dirichlet barrier that localizes the ground state and justifies truncation/imaginary-time evolution in polynomial time.}

% =========================
% Optional: add bib entries (if you have a bibliography section elsewhere, omit these)
% =========================
% \begin{thebibliography}{9}
% \bibitem{CLR}
% P.~Caputo, T.~Liggett, and T.~Richthammer.
% \newblock Proof of Aldous' spectral gap conjecture.
% \newblock \emph{J. Amer. Math. Soc.} 23 (2010), 831--851. arXiv:0906.1238.
%
% \bibitem{CesiOctopus}
% F.~Cesi.
% \newblock On the octopus inequality.
% \newblock arXiv:1310.6156.
% \end{thebibliography}


\pagebreak 

% ============================================================
% Drop-in Roadmap (Aldous/CLR -> Octopus Proof Ideas -> Transition)
% Clear, concise, presentation-ready narrative + pointers.
% ============================================================

\section{Roadmap: From Aldous/CLR to Our New Localization Ideas}

\subsection{Situation setup: why Aldous matters for our move graph}

\paragraph{Base object: a weighted graph.}
Fix a finite vertex set $V$ with $|V|=n$ and symmetric edge rates $c_{xy}=c_{yx}\ge 0$.

\paragraph{Two Markov chains built from the same edge rates.}
\begin{itemize}[leftmargin=1.6em,itemsep=0.25em]
\item \textbf{Random walk (RW)} on $V$:
\[
(L_{\rw} f)(x)=\sum_{y\neq x} c_{xy}\,\bigl(f(y)-f(x)\bigr).
\qquad
\lambda^{\rw}(G):=\lambda_1(L_{\rw}).
\]
\item \textbf{Interchange process (IP)} on permutations $\mathfrak{S}_n$:
\[
(L_{\ip}F)(\eta)=\sum_{x<y} c_{xy}\,\bigl(F(\eta\tau_{xy})-F(\eta)\bigr),
\qquad
\lambda^{\ip}(G):=\lambda_1(L_{\ip}),
\]
where $\tau_{xy}$ swaps the labels at $x,y$ and $\nu$ is uniform on $\mathfrak{S}_n$.
\end{itemize}

\paragraph{Proof interface: Dirichlet forms.}
For a reversible chain $(\Omega,\pi,L)$,
\[
\mathcal{E}_L(f)=\frac12\sum_{\omega,\omega'}\pi(\omega)q(\omega,\omega')\bigl(f(\omega')-f(\omega)\bigr)^2,
\qquad
\lambda_1(L)=\inf_{f:\Var_\pi(f)\neq 0}\frac{\mathcal{E}_L(f)}{\Var_\pi(f)}.
\]
For IP:
\[
\mathcal{E}_{\ip}(F)=\frac12\sum_{a<b} c_{ab}\,\nu\!\left[\bigl(\nabla_{ab}F\bigr)^2\right],
\qquad
\nabla_{ab}F(\eta)=F(\eta\tau_{ab})-F(\eta).
\]

\paragraph{Easy direction (factor map).}
Tracking one label under IP yields RW as a factor, hence
\[
\lambda^{\ip}(G)\le \lambda^{\rw}(G).
\]
Thus the substantive task is $\lambda^{\ip}(G)\ge \lambda^{\rw}(G)$.

\paragraph{CLR theorem (Aldous' conjecture).}
\[
\boxed{\ \lambda^{\ip}(G)=\lambda^{\rw}(G)\ \ \text{for all weighted graphs }G.\ }
\]
\emph{Takeaway for expanders:} if $G$ is a bounded-degree expander with constant RW gap, then the many-body interchange move graph inherits a constant gap, i.e.\ it introduces no additional bottleneck beyond the base graph.

\medskip
\noindent\textbf{Where in CLR:} Theorem~2.3 is the octopus inequality; its proof is Section~3. The main theorem is proved by elimination + induction using octopus.

\subsection{Key idea enabling induction: elimination and the octopus inequality}

\paragraph{Vertex elimination (RW).}
Fix $x\in V$, let $V_x:=V\setminus\{x\}$ and $c_x:=\sum_{w\in V_x}c_{xw}$.
Define Kron weights on $V_x$:
\[
c^{*,x}_{yz}:=\frac{c_{xy}c_{xz}}{c_x}\qquad (y\neq z\in V_x),
\]
and reduced graph $G_x$ with $\tilde c_{yz}:=c_{yz}+c^{*,x}_{yz}$.
A classical monotonicity is
\[
\lambda^{\rw}(G_x)\ge \lambda^{\rw}(G).
\]
\emph{Moral:} RW has literal star--mesh reduction.

\paragraph{What we need for IP.}
To induct on $n$ for IP, we need an \emph{IP analogue} of star--mesh:
a Dirichlet-form domination that replaces the star at $x$ by the reduced clique on $V_x$ with the \emph{same} Kron coefficients.

\paragraph{Octopus inequality (IP Dirichlet domination).}
For every $F:\mathfrak{S}_n\to\mathbb{R}$ and every $x\in V$,
\[
\boxed{\ \sum_{y\in V_x} c_{xy}\,\nu\!\left[\bigl(\nabla_{xy}F\bigr)^2\right]
\ \ge\
\sum_{\substack{y<z\\ y,z\in V_x}} c^{*,x}_{yz}\,\nu\!\left[\bigl(\nabla_{yz}F\bigr)^2\right]\ }.
\]
\emph{Interpretation:} ``tentacles'' (swaps touching $x$) dominate the reduced clique swaps among $V_x$.

\medskip
\noindent\textbf{Where in CLR:} Octopus is Theorem~2.3 / (2.4); the proof is Section~3.

\subsection{How octopus is proved in CLR: the 6-step proof arc (presentation summary)}

Throughout CLR's Section~3, relabel $V=\{0,1,\dots,n-1\}$ and fix the hub $x=0$.
Write $c_i:=c_{0i}$ for $i\ge 1$ and set $c_0:=-\sum_{i\ge 1}c_i<0$.

\begin{enumerate}[leftmargin=1.8em,itemsep=0.35em]
\item \textbf{Rewrite octopus as a single quadratic form inequality.}
CLR rewrite (2.4) into (3.2), equivalent to nonnegativity of a single sum of squared differences over transpositions.
\emph{(CLR: start of Section~3; equation (3.2)).}

\item \textbf{Encode as a PSD matrix problem.}
Define an explicit matrix $C$ indexed by permutations (equation (3.3)) so that (3.2) holds for all $f$ iff $C\succeq 0$.
\emph{(CLR: equation (3.3)).}

\item \textbf{Parity block decomposition $\Rightarrow$ reduce to a Schur complement.}
Since transpositions flip parity, order permutations as (even, odd) to get
$C=\bigl(\begin{smallmatrix}cI & X^{\mathsf T}\\ X & cI\end{smallmatrix}\bigr)$.
Subtract a manifestly PSD Gram matrix $\widetilde C=A^{\mathsf T}A$ to reduce $C\succeq 0$ to
\[
C'(n):=c^2 I - X^{\mathsf T}X \succeq 0.
\]
\emph{(CLR: Subsection~3.1; equation (3.4)).}

\item \textbf{Expand the correction matrix into universal 4-vertex pieces.}
Lemma~3.1 gives
\[
C'(n)=\sum_{J\subset V:\ |J|=4} (-c_J)\,A_J(n),\qquad c_J=\prod_{i\in J}c_i,
\]
where $A_J(n)$ is rate-free (depends only on the 4-set $J$).
\emph{(CLR: Lemma~3.1).}

\item \textbf{Show each $A_J(n)$ is PSD (local block reduction).}
Lemma~3.2 shows $A_J(n)$ is block diagonal, each block a copy of a fixed $12\times 12$ matrix,
whose spectrum is $\{0,12\}$, hence $A_J(n)\succeq 0$.
\emph{(CLR: Lemma~3.2; discussion around (3.8)).}

\item \textbf{Resolve mixed signs by bundling into PSD 5-vertex packages.}
Because coefficients $(-c_J)$ can have either sign (due to $c_0<0$),
CLR define for each 5-set $K\ni 0$ a signed combination
$B_K(n)=\sum_{J\subset K, |J|=4}\varepsilon_J A_J(n)$ and prove $B_K(n)\succeq 0$ (Lemma~3.3),
reducing to a single canonical $60\times60$ check and showing it is a scaled projection ($B^2=24B$).
Then Lemma~3.4 gives a coefficient ``sandwich'':
\[
0\ \le\ \sum_K \frac{|c_K|}{|c_0|}B_K(n)\ \le\ C'(n),
\]
implying $C'(n)\succeq 0$, hence $C\succeq 0$, hence octopus.
\emph{(CLR: Lemma~3.3 and Lemma~3.4 / Subsection~3.3).}
\end{enumerate}

\paragraph{One-line message.}
\emph{Octopus is proved by turning the Dirichlet comparison into an explicit PSD matrix statement and reducing it,
via parity blocks and local (4- and 5-vertex) block decompositions, to a finite ``small matrix'' positivity verification.}

\subsection{Closing Aldous from octopus (one-slide proof spine)}

\begin{enumerate}[leftmargin=1.8em,itemsep=0.25em]
\item RW elimination monotonicity: $\lambda^{\rw}(G_x)\ge \lambda^{\rw}(G)$.
\item Octopus: the IP Dirichlet form on the star at $x$ dominates the IP Dirichlet form on the reduced clique (with Kron weights).
\item Induct on $|V|$: apply the induction hypothesis on $G_x$ and combine with elimination to obtain
$\lambda^{\ip}(G)\ge \lambda^{\rw}(G)$.
\item Together with $\lambda^{\ip}(G)\le \lambda^{\rw}(G)$, conclude equality.
\end{enumerate}

\subsection{Transition: what Aldous gives us, and what must be new for our algorithm}

\paragraph{What Aldous/CLR gives (template).}
A \emph{spectral-geometry template} for large configuration spaces:
local generators $\Rightarrow$ Dirichlet-form domination (octopus) $\Rightarrow$ elimination/induction $\Rightarrow$ constant gap on expanders.

\paragraph{What changes in our setting (new ideas).}
In our move-graph/QMC setting we still seek \emph{local-to-global control}, but:
\begin{itemize}[leftmargin=1.6em,itemsep=0.25em]
\item the state space is not $\mathfrak{S}_n$ (it is a constrained combinatorial space, e.g.\ perfect matchings / VB coverings),
\item moves are not arbitrary transpositions (they are local $4$-cycle flips),
\item and the effective dynamics typically includes a \emph{potential well} (diagonal energy) and/or non-uniform weights.
\end{itemize}
Thus the next step is to build the \emph{analogue} of CLR's Dirichlet-form control in this new configuration space:
we will prove a \emph{Dirichlet barrier / localization} statement that justifies truncating to a constant-radius neighborhood
and running imaginary-time evolution inside that truncated subspace.

\paragraph{Hand-off sentence (say this out loud).}
\emph{Aldous/CLR shows that on an expander, the pure kinetic move graph has constant gap via a sharp Dirichlet-form inequality (octopus).
Our contribution is to combine this kinetic expansion with the diagonal potential to prove localization (a Dirichlet barrier),
which yields a polynomial-time algorithm by truncation plus imaginary-time evolution.}

% ============================================================
% End Roadmap
% ============================================================

\vspace{-0.4em}
\bibliographystyle{plain}
\begin{thebibliography}{9}

\bibitem{CLR}
P.~Caputo, T.~Liggett, and T.~Richthammer.
\newblock {Proof of Aldous' spectral gap conjecture}.
\newblock \emph{Journal of the American Mathematical Society} 23 (2010), 831--851.
\newblock Preprint: \href{https://arxiv.org/abs/0906.1238}{arXiv:0906.1238}.

\end{thebibliography}

\end{document}