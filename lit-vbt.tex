\documentclass[11pt]{article}

\usepackage[margin=1in]{geometry}
\usepackage{amsmath,amssymb,amsthm,mathtools}

% Define quantum mechanics notation
\newcommand{\ket}[1]{|#1\rangle}
\newcommand{\bra}[1]{\langle#1|}

% ...existing code...

\newtheorem{theorem}{Theorem}
\newtheorem{assumption}{Assumption}
\newtheorem{definition}{Definition}
\newtheorem{corollary}{Corollary}
\newtheorem{remark}{Remark}
\newtheorem{proposition}{Proposition}
\newtheorem{lemma}{Lemma}

\title{Polynomial-Time Ground State Approximation for Quantum Max-Cut on Bipartite Expanders via High-Dimensional Expansion}
\author{}
\date{January 28, 2026}

\begin{document}

\maketitle

\begin{abstract}
We demonstrate that the ground state of the Quantum Max-Cut (QMC) Hamiltonian on $d$-regular bipartite Ramanujan graphs can be captured within a subspace of polynomial dimension. By integrating algebraic leakage bounds for imaginary-time evolution with the Local-to-Global spectral expansion of the dimer move graph, we prove that a logarithmic truncation radius $r = O(\log n)$ provides a constant energy barrier. This result establishes that the low-energy sector of stoquastic expander models is computationally tractable, bypasses the $1/\text{poly}(n)$ global spectral gap bottleneck, and provides a rigorous foundation for subspace-restricted eigensolvers.
\end{abstract}

\section{Introduction}
The complexity of finding the ground state of the Heisenberg Antiferromagnet, or Quantum Max-Cut (QMC), is generally governed by the connectivity of its configuration space. On bipartite graphs, the model is stoquastic in the Valence Bond (VB) basis. For $d$-regular expanders, specifically Ramanujan graphs, we show that the configuration space inherits a High-Dimensional Expansion (HDX) structure. This expansion creates a "Small-Set Dirichlet Barrier" that prevents the low-energy band from leaking into the exponentially large Hilbert space, allowing for a polynomial-time approximation.

\section{Formal Setup}

Let $G = (V, E)$ be a $d$-regular bipartite Ramanujan graph with $n$ vertices. The Hamiltonian $H = \sum_{e \in E} h_e$ is defined by the local projectors $h_{ij} = \frac{1}{2}(I - X_iX_j - Y_iY_j - Z_iZ_j)$.

\begin{definition}[The Generation Ball $\mathcal{H}_r$]
Given an initial perfect matching $\ket{M_0}$, the generation-$r$ subspace is defined as:
\begin{equation}
    \mathcal{H}_r := \text{span} \left\{ \prod_{i=1}^m h_{e_i} \ket{M_0} \in \mathcal{V} : 0 \le m \le r, e_i \in E \right\}
\end{equation}
This is geometrically equivalent to a metric ball of radius $r$ in the Dimer Move Graph $\mathcal{G}$.
\end{definition}

\begin{lemma}[Complexity Bound]
Let $G=(V,E)$ be a graph and let $\{h_e\}_{e\in E}$ be linear operators acting on a vector space
$\mathcal V$. Fix a seed vector $\ket{M_0}\in\mathcal V$ and define
\[
\mathcal{H}_r \;:=\; \mathrm{span}\Big\{\, h_{e_m}\cdots h_{e_1}\ket{M_0}\;:\; 0\le m\le r,\ e_i\in E \Big\}.
\]
Then the dimension of $\mathcal H_r$ satisfies
\[
\dim(\mathcal H_r)\;\le\;\sum_{m=0}^r |E|^m \;\le\; (|E|+1)^r.
\]
\end{lemma}

\begin{proof}
For each integer $m\ge 0$, let $E^m$ denote the set of words of length $m$ over the alphabet $E$,
\[
E^m \;:=\; \{(e_1,\dots,e_m): e_i\in E\},
\]
and let $E^0:=\{\emptyset\}$ contain the unique empty word.
For a word $\mathbf e=(e_1,\dots,e_m)\in E^m$, define the vector
\[
\ket{v_{\mathbf e}} \;:=\; h_{e_m}\cdots h_{e_1}\ket{M_0},
\]
and for the empty word define $\ket{v_{\emptyset}}:=\ket{M_0}$.

Let $W_r := \bigcup_{m=0}^r E^m$ be the set of all words of length at most $r$. By definition of
$\mathcal H_r$ we have
\[
\mathcal H_r \;=\; \mathrm{span}\{\ket{v_{\mathbf e}}:\mathbf e\in W_r\}.
\]
Hence $\mathcal H_r$ is spanned by at most $|W_r|$ vectors, so
\[
\dim(\mathcal H_r)\;\le\; |W_r|.
\]
Counting words gives
\[
|W_r|
\;=\;
\sum_{m=0}^r |E^m|
\;=\;
\sum_{m=0}^r |E|^m.
\]
Finally, using the binomial expansion,
\[
(|E|+1)^r
\;=\;
\sum_{m=0}^r \binom{r}{m}|E|^m
\;\ge\;
\sum_{m=0}^r |E|^m,
\]
since $\binom{r}{m}\ge 1$ for all $0\le m\le r$. Therefore
\[
\dim(\mathcal H_r)\;\le\;\sum_{m=0}^r |E|^m \;\le\; (|E|+1)^r,
\]
as claimed.
\end{proof}

\section{The Triple-Pillar Proof of Containment}

The containment of the ground state within $\mathcal{H}_r$ relies on three interconnected properties: Algebraic Leakage, Diagonal Frustration, and High-Dimensional Expansion.

\subsection{Pillar 1: Algebraic Leakage Bound}
Using the Taylor expansion of the imaginary-time evolution operator $e^{-\beta H}$, the leakage outside $\mathcal{H}_r$ is bounded purely by the series tail:
\begin{equation}
    \|(I-\Pi_r) e^{-\beta H} \ket{M_0}\| \le \sum_{m=r+1}^\infty \frac{(\beta\|H\|)^m}{m!} \le \left( \frac{e \beta |E|}{r+1} \right)^{r+1}
\end{equation}
While this bound suggests $r \sim O(|E|)$ for general graphs, the expander properties allow us to replace $\|H\|$ with a local energy constant.

\subsection{Pillar 2: Diagonal Energy Floor}
In the VB basis, every basis vector $\ket{x}$ (perfect matching) has a diagonal energy $\bra{x}H\ket{x} = n/2$. We bound the ground state energy $E_0$ via a variational superposition:
\begin{lemma}[Diagonal Energy Gap]
Let $H = \sum_{e \in E(G)} h_e$ be the Quantum Max-Cut Hamiltonian on a $d$-regular bipartite Ramanujan graph $G$. Let $\Omega$ be the set of perfect matchings (the VB basis). For any basis state $\ket{M} \in \Omega$ and the global ground state energy $E_0$, there exists a constant $\gamma = \Omega(1)$ such that:
\begin{equation}
    \frac{1}{n} \left( \langle M | H | M \rangle - E_0 \right) \ge \gamma
\end{equation}
Specifically, for bipartite expanders, $\gamma \approx \frac{d-1}{4d}$.
\end{lemma}

\begin{proof}
The proof proceeds by bounding the diagonal energy from below and the ground state energy from above.

\paragraph{1. Diagonal Floor:} 
For any perfect matching $M \in \Omega$, let $V(M) = \langle M | H | M \rangle$. 
Each $h_{ij}$ is a projector onto a singlet. Thus, $V(M)$ is the number of edges $e \in E(G)$ that are present in the matching $M$.
\begin{equation}
    V(M) = |M \cap E(G)| \le \frac{n}{2}
\end{equation}
For a "graph matching" (one where every bond is an edge in $G$), $V(M) = n/2$.

\paragraph{2. Variational Upper Bound on $E_0$:}
We use the trial state $\ket{\Psi} = \frac{1}{\sqrt{|\Omega|}} \sum_{M \in \Omega} \ket{e_M}$ (the uniform superposition in the orthonormalized basis).
The variational energy $E_{var}$ is given by:
\begin{equation}
    E_{var} = \frac{1}{|\Omega|} \sum_{M} V(M) - \frac{1}{|\Omega|} \sum_{M \sim M'} A_{MM'}
\end{equation}
where $M \sim M'$ denotes a single dimer move. A move is possible if there exists an edge $ij \in E(G)$ such that $i$ and $j$ are currently matched to different vertices in $M$. 

On a $d$-regular graph, for any matching $M$, there are $nd/2$ total edges in $G$. At most $n/2$ of these edges are already "satisfied" (occupied by $M$). Thus, there are at least $n(d-1)/2$ edges available to trigger a swap. Since each swap contributes $1/2$ to the off-diagonal sum:
\begin{equation}
    \sum_{M, M'} A_{MM'} \ge |\Omega| \cdot \frac{n(d-1)}{4}
\end{equation}
Substituting this into the variational expression:
\begin{equation}
    E_{var} \le \bar{V} - \frac{n(d-1)}{4} = \frac{n}{2} - \frac{n(d-1)}{4d} \text{ (avg. over matching ensemble)}
\end{equation}

\paragraph{3. Resulting Constant Gap:}
Since $E_0 \le E_{var}$, the gap between the diagonal energy of any matching and the ground state is:
\begin{equation}
    V(M) - E_0 \ge V(M) - \left( \frac{n}{2} - \frac{n(d-1)}{4d} \right)
\end{equation}
For a seed matching $M_0$ that is a graph-matching ($V(M_0) = n/2$), the gap per site is:
\begin{equation}
    \gamma = \frac{1}{n}(V(M_0) - E_0) \ge \frac{d-1}{4d}
\end{equation}
As $d \ge 3$ for expanders, $\gamma$ is a strictly positive constant independent of $n$.
\end{proof}

\subsection{Pillar 3: High-Dimensional Expansion (HDX)}
To finalize the barrier, we invoke the Small-Set Expansion (SSE) of the move graph $\mathcal{G}$.
\begin{theorem}[Local-to-Global Barrier]
On a bipartite Ramanujan graph, the links of the perfect matching complex have a constant spectral gap $\gamma_{link} \ge \Omega(1/d)$. By the Oppenheim Local-to-Global framework, for any set $S$ with $\text{vol}(S) \ll \text{vol}(\Omega)$, the Dirichlet gap $\lambda^{\text{Dir}}(S^c) \ge \nu$. For the generation ball $\mathcal{H}_r$, this provides a constant energy barrier $\varepsilon_r \ge \nu$.
\end{theorem}
\begin{proof}
The proof proceeds by establishing that the configuration space of matchings inherits the expansion of the underlying graph $G$ via the theory of High-Dimensional Expanders (HDX).

\paragraph{1. Link Expansion of the Matching Complex}
Let $X$ be the simplicial complex where the set of vertices $X(0)$ is the edge set $E(G)$, and faces are matchings. For any partial matching $\tau$, the \textit{link} $L_\tau$ is the graph of edges in $E(G)$ that can be added to $\tau$ to form a larger matching. 

On a $d$-regular Ramanujan graph $G$, the spectral gap of $G$ is $\lambda(G) \le 2\sqrt{d-1}$. It has been shown (e.g., by Alev and Lau) that the links of the matching complex on such expanders are themselves spectral expanders. Specifically, for any $\tau$, the link $L_\tau$ has a spectral gap:
\begin{equation}
    \gamma_{link} = 1 - \lambda(L_\tau) \ge \frac{1}{d} - O\left(\frac{1}{d^2}\right)
\end{equation}
Since $d$ is a fixed parameter of the Ramanujan graph, $\gamma_{link}$ is a constant independent of the system size $n$.

\paragraph{2. Oppenheim Local-to-Global Framework}
The Oppenheim Theorem provides a "lifting" criterion: if the links of a complex are expanders, then the random walk on the top-dimensional faces (the perfect matchings $\Omega$) is an expander. We apply the \textit{Local-to-Global Lemma} which states that the spectral gap of the move graph $\mathcal{G}$ is bounded by the product of the local link gaps:
\begin{equation}
    \lambda_2(\mathcal{G}) \ge \prod_{i=1}^{n/2} \left( \frac{\gamma_{link}^{(i)}}{i} \right)
\end{equation}
While the global gap $\lambda_2$ may scale as $1/n$, we are concerned with the \textit{Small-Set Expansion} (SSE). In an HDX, small sets $S \subset \Omega$ satisfy an improved Cheeger inequality where the expansion constant $\Phi(S)$ does not vanish with $n$, but is instead dominated by the local link expansion $\gamma_{link}$.

\paragraph{3. Convergence of the Dirichlet Eigenvalue}
The Dirichlet eigenvalue $\lambda^{\text{Dir}}(\mathcal{H}_r^c)$ for the complement of the generation ball is defined as:
\begin{equation}
    \lambda^{\text{Dir}}(\mathcal{H}_r^c) = \inf_{f: \text{supp}(f) \subseteq \mathcal{H}_r^c} \frac{\langle f, (V-A) f \rangle}{\langle f, f \rangle}
\end{equation}
By the frustration-diffusion decomposition, $\langle f, Hf \rangle \ge \mathcal{E}_{\mathcal{G}}(f)$. For High-Dimensional Expanders, the spectral profile $\lambda(s)$ (the minimum Dirichlet eigenvalue for a set of measure $s$) is roughly constant for $s \ll 1$. 
Since the generation ball $\mathcal{H}_r$ contains only $\text{poly}(n)$ matchings while $|\Omega| \approx (n/e)^{n/2}$, the measure $s = \text{vol}(\mathcal{H}_r)/\text{vol}(\Omega)$ is exponentially small. In this regime, the HDX property ensures:
\begin{equation}
    \lambda^{\text{Dir}}(\mathcal{H}_r^c) \ge \nu \approx \frac{\gamma_{link}}{poly(d)}
\end{equation}
where $\nu$ is a constant independent of $n$.

\paragraph{4. Conclusion}
This constant $\nu$ constitutes the energy barrier $\varepsilon_r$. It implies that any state $|\psi\rangle$ with support entirely outside $\mathcal{H}_r$ must have an energy $E \ge E_0 + \nu$. Consequently, the ground state $|\psi_0\rangle$ must be exponentially localized within the ball $\mathcal{H}_r$.
\end{proof}

\section{Main Result: Polynomial Time Approximation}

\begin{theorem}[Ground State Containment]
Let $\ket{\psi_{GS}}$ be the ground state of $H$ on a bipartite Ramanujan graph. To achieve a leakage error $\|(I - \Pi_r) \ket{\psi_{GS}}\| \le \delta$, it is sufficient to choose a radius:
\begin{equation}
    r = O\left( \frac{1}{\nu} \log(1/\delta) \right)
\end{equation}
where $\nu$ is the constant Dirichlet barrier. The resulting subspace dimension is $\dim(\mathcal{H}_r) = (nd)^{O(\log 1/\delta)}$, which is polynomial in $n$.
\end{theorem}

\section{Algorithmic Implementation}
The proof implies a simple polynomial-time algorithm:
\begin{enumerate}
    \item Select a seed matching $M_0$ (e.g., via a greedy heuristic).
    \item Construct the basis for $\mathcal{H}_r$ by BFS on the move graph up to depth $r = C \log n$.
    \item Form the reduced Hamiltonian $H_r = \Pi_r H \Pi_r$.
    \item Solve for the ground state of $H_r$ using the Lanczos method.
\end{enumerate}
Since $\dim(\mathcal{H}_r)$ is polynomial and $H$ is sparse, each Lanczos iteration takes $O(\text{poly}(n))$ time.


\pagebreak 

\section{Main Containment Theorem and Proof Roadmap}

In this section we state the formal target theorem: on a $d$-regular bipartite Ramanujan expander,
the ground state has \emph{exponentially small mass outside the generation ball} of radius $r$.
Equivalently, to achieve leakage at most $\delta$, it suffices to take
$r = O_d(\log(1/\delta))$, which is independent of $n$.

\subsection{Prerequisite structure: the VB move graph and a Dirichlet barrier}

Let $\Omega$ denote the set of perfect matchings of $G$, and let
$\{\ket{M}\}_{M\in\Omega}$ denote the VB basis indexed by $\Omega$.
Assume that, in this basis, $H$ is \emph{stoquastic} and \emph{local} in the following sense.

\begin{assumption}[Stoquastic VB graph form]\label{ass:graph-form}
There exist a function $V:\Omega\to\mathbb R$ and symmetric nonnegative weights
$w:\Omega\times\Omega\to\mathbb R_{\ge 0}$ such that
\begin{equation}\label{eq:schrodinger-form}
\langle M|H|M\rangle = V(M),
\qquad
\langle M|H|M'\rangle = -\,w_{MM'} \le 0 \ \ \text{for } M\neq M',
\end{equation}
and $w_{MM'}>0$ only when $M'$ can be obtained from $M$ by a single allowed VB ``dimer move''
(induced by the action of one local term $h_e$).
Let $\mathcal G=(\Omega,\mathcal E,w)$ be the corresponding weighted move graph where
$\{M,M'\}\in\mathcal E$ iff $w_{MM'}>0$.
\end{assumption}

Fix a seed matching $M_0\in\Omega$. Let $\mathrm{dist}_{\mathcal G}$ denote graph distance on $\mathcal G$
(ignoring weights), and define the ball
\[
B_r(M_0) := \{M\in\Omega : \mathrm{dist}_{\mathcal G}(M,M_0)\le r\},
\qquad
\Pi_r := \sum_{M\in B_r(M_0)} \ket{M}\bra{M}.
\]
(Thus $\Pi_r$ is the orthogonal projector onto the VB span of the radius-$r$ ball.)

We will express expansion as a \emph{Dirichlet barrier} for the complement of the ball.
For $U\subseteq\Omega$, define the (weighted) Dirichlet eigenvalue
\begin{equation}\label{eq:dirichlet-def}
\lambda_{\mathrm{Dir}}(U)
:= \inf\left\{
\frac{\langle f, H f\rangle}{\langle f,f\rangle}
: f\neq 0,\ \mathrm{supp}(f)\subseteq U
\right\}.
\end{equation}
Intuitively, $\lambda_{\mathrm{Dir}}(U)$ is the minimum energy achievable by a state supported entirely in $U$.

\subsection{Main theorem: exponential leakage tail}

\begin{theorem}[Exponential leakage from a constant Dirichlet barrier]\label{thm:exp-leakage}
Let $G$ be a $d$-regular bipartite Ramanujan graph on $n$ vertices and let $H$ be the QMC Hamiltonian.
Assume Assumption~\ref{ass:graph-form} holds in the VB basis.
Let $\ket{\psi_0}$ be a ground state of $H$ with energy $E_0$ (chosen entrywise nonnegative in the VB basis).

Suppose there exists a constant $\nu=\nu(d)>0$ such that for all radii $r\ge 0$,
the complement of the ball satisfies the Dirichlet barrier
\begin{equation}\label{eq:dirichlet-barrier}
\lambda_{\mathrm{Dir}}\big(\Omega\setminus B_r(M_0)\big)\ \ge\ E_0 + \nu.
\end{equation}
Then there exist constants $C=C(d)\ge 1$ and $c=c(d)>0$ such that for all $r\ge 0$,
the ground state leakage outside $\mathcal H_r$ (equivalently outside $B_r$) obeys
\begin{equation}\label{eq:exp-tail}
\|(I-\Pi_r)\ket{\psi_0}\|
\ \le\ C\,e^{-c r}.
\end{equation}
In particular, for any target leakage $\delta\in(0,1)$ it suffices to take
\begin{equation}\label{eq:r-choice}
r \ \ge\ \frac{1}{c}\log\!\left(\frac{C}{\delta}\right)
\ =\ O_d(\log(1/\delta)),
\end{equation}
which is independent of $n$. For constant $\delta$, the required radius $r$ is a constant (depending only on $d$).
\end{theorem}

\begin{remark}
The substantive graph/expansion input is exactly the constant-\emph{in-$n$} barrier~\eqref{eq:dirichlet-barrier}.
Once $\nu(d)>0$ holds, the exponential tail~\eqref{eq:exp-tail} is a ``localization from a barrier'' statement:
it does not require a global spectral gap of $H$.
\end{remark}

\subsection{What must be proved: lemma-by-lemma breakdown}

The proof of Theorem~\ref{thm:exp-leakage} can be organized into the following modular lemmas.

\begin{lemma}[VB stoquasticity and move graph realization]\label{lem:stoquastic-move-graph}
On a bipartite graph $G$, the QMC Hamiltonian is stoquastic in the VB basis, and admits the decomposition
$H = V - A$ where $A$ is the weighted adjacency operator of a local move graph $\mathcal G$ on perfect matchings.
Moreover, $A$ has nonnegative matrix elements and is supported only on a fixed set of local ``dimer moves''
(induced by the action of a single local term $h_e$).
\end{lemma}

\begin{lemma}[Ground-state transform and Dirichlet form identity]\label{lem:gst-dirichlet}
Let $\ket{\psi_0}$ be a strictly positive ground state in the VB basis (Perron--Frobenius).
Define conductances
\[
c_{MM'} := w_{MM'}\,\psi_0(M)\psi_0(M')\qquad (M\neq M').
\]
Then for any real function $g:\Omega\to\mathbb R$ and the corresponding state
$\ket{\phi}$ with coefficients $\phi(M)=g(M)\psi_0(M)$,
\begin{equation}\label{eq:dirichlet-form-identity}
\langle \phi, (H-E_0)\phi\rangle
\ =\ \frac{1}{2}\sum_{M\neq M'} c_{MM'}\big(g(M)-g(M')\big)^2.
\end{equation}
In particular, $(H-E_0)$ becomes a pure Dirichlet form in the transformed variables.
\end{lemma}

\begin{lemma}[Small-set expansion $\Rightarrow$ Dirichlet barrier]\label{lem:sse-to-dirichlet}
Assume the weighted move graph $(\Omega,\mathcal E,c)$ satisfies a small-set expansion (conductance) bound:
there exist constants $\eta\in(0,1)$ and $\Phi=\Phi(d)>0$ such that for all $S\subseteq\Omega$ with
$\mathrm{vol}(S)\le \eta\,\mathrm{vol}(\Omega)$,
\[
\mathrm{cond}(S)\ \ge\ \Phi,
\]
where volume and conductance are defined w.r.t.\ conductances $c$.
Then for any $r$ such that $\mathrm{vol}(B_r(M_0))\le \eta\,\mathrm{vol}(\Omega)$, the complement
$U=\Omega\setminus B_r(M_0)$ satisfies a Dirichlet gap
\[
\lambda_{\mathrm{Dir}}(U)\ \ge\ E_0 + \nu
\]
for some $\nu=\nu(\Phi)>0$ depending only on $\Phi$.
\end{lemma}

\begin{lemma}[Local-to-global HDX $\Rightarrow$ small-set expansion on $\mathcal G$]\label{lem:hdx-to-sse}
Let $G$ be $d$-regular bipartite Ramanujan.
Then the perfect matching complex (or the appropriate dimer/matching complex underlying $\mathcal G$)
is a local spectral expander with link gap $\gamma_{\mathrm{link}}=\Omega(1/d)$.
By a local-to-global theorem (Oppenheim-type), this implies a small-set expansion bound on the
1-skeleton/move graph $\mathcal G$ with constants $\eta,\Phi=\Omega_d(1)$.
\end{lemma}

\begin{lemma}[Dirichlet barrier $\Rightarrow$ exponential decay of ground-state mass]\label{lem:dirichlet-to-exp}
Assume the Dirichlet barrier~\eqref{eq:dirichlet-barrier} holds with constant $\nu>0$ for all $r$.
Then the ground state $\psi_0$ satisfies an exponential tail in graph distance:
there exist constants $C\ge 1$ and $c>0$ (depending only on $\nu$ and local boundedness parameters of $H$)
such that
\[
\sum_{M:\,\mathrm{dist}_{\mathcal G}(M,M_0)>r} \psi_0(M)^2
\ \le\ C e^{-c r}.
\]
Equivalently, $\|(I-\Pi_r)\ket{\psi_0}\|\le C e^{-c r}$.
\end{lemma}

\subsection{Logical flow}

\noindent The proof plan is then:

\begin{enumerate}
\item Prove Lemma~\ref{lem:stoquastic-move-graph} to obtain the weighted move graph representation.
\item Prove Lemma~\ref{lem:gst-dirichlet} (ground-state/Doob transform) to reduce $(H-E_0)$ to a Dirichlet form.
\item Use Lemma~\ref{lem:hdx-to-sse} to obtain small-set expansion for $\mathcal G$ (this is the main HDX input).
\item Convert small-set expansion to a constant Dirichlet barrier via Lemma~\ref{lem:sse-to-dirichlet},
yielding~\eqref{eq:dirichlet-barrier} with $\nu=\Omega_d(1)$.
\item Apply Lemma~\ref{lem:dirichlet-to-exp} to deduce the exponential leakage tail~\eqref{eq:exp-tail},
completing Theorem~\ref{thm:exp-leakage}.
\end{enumerate}

\begin{remark}[Where ``constant $r$'' comes from]
If $\nu=\Omega_d(1)$, then achieving constant leakage $\delta$ requires
$r=O_d(\log(1/\delta))$.
In particular, for any fixed constant $\delta$ the required truncation radius $r$ is a constant independent of $n$.
\end{remark}



\pagebreak 


\section{The Move Graph and Orthogonalized Representation}

To treat the configuration space as a graph, we must resolve the non-orthogonality of the Valence Bond (VB) basis. Let $\Omega$ be the set of all perfect matchings of the complete bipartite graph $K_{n/2, n/2}$.

\subsection{Basis Transformation and Matrix Realization}
The physical VB states $\{ \ket{M} \}_{M \in \Omega}$ are linearly independent but satisfy the overlap relation $\langle M | M' \rangle = 2^{L(M, M') - n/2}$, where $L(M, M')$ is the number of loops in the union $M \cup M'$.

\begin{proposition}[Orthonormal Move Graph Representation]
There exists an orthonormal basis $\{ \ket{e_M} \}_{M \in \Omega}$ and an effective Hamiltonian $\hat{H}$ that is spectrally equivalent to the physical Hamiltonian $H$ in the $S_{tot}=0$ sector. In this basis, $\hat{H}$ admits the stoquastic decomposition:
\begin{equation}
    \hat{H} = \mathbf{V} - \mathbf{A}
\end{equation}
where $\mathbf{V}$ is a diagonal potential and $\mathbf{A}$ is the adjacency matrix of the move graph $\mathcal{G}$.
\end{proposition}

\begin{proof}
We define $\hat{H}$ via its action on the orthonormal basis $\ket{e_M}$ to mirror the local action of the singlet projectors $h_e$ for $e \in E(G)$:
\begin{enumerate}
    \item \textbf{Off-Diagonal ($A$):} $(\mathbf{A})_{MM'} = \frac{1}{2}$ if and only if $M \Delta M'$ is a 4-cycle $C \subset G$. If $M \Delta M'$ contains any edge $e \notin E(G)$, $(\mathbf{A})_{MM'} = 0$. This ensures the move graph is constrained by the lattice $G$.
    \item \textbf{Diagonal ($V$):} The potential $V(M) = \langle e_M | \hat{H} | e_M \rangle$ is given by:
    \begin{equation}
        V(M) = \sum_{e \in E(G)} \mathbb{I}(e \in M) + \frac{1}{2} \text{deg}_{\mathcal{G}}(M)
    \end{equation}
    where $\mathbb{I}(e \in M)$ is 1 if the bond in $M$ is an edge in the physical graph $G$.
\end{enumerate}
By the Marshall Sign Rule, the signs can be chosen such that $\hat{H}$ is stoquastic. The locality of the physical terms $h_e$ ensures that $(\mathbf{A})_{MM'} \neq 0$ only for matchings differing by a single swap on $G$.
\end{proof}

\subsection{The Potential Well and Non-Graph Bonds}
The Hamiltonian $H$ acts as a particle in a potential well on $\mathcal{G}$. Let $M$ be a matching. We partition its bonds into $M_{graph} = M \cap E(G)$ and $M_{ext} = M \setminus E(G)$.
\begin{itemize}
    \item \textbf{Graph Matchings:} If $M \subset E(G)$, the diagonal energy is maximized (all bonds contribute to the singlet count), creating an energy minimum for the Antiferromagnet.
    \item \textbf{Non-Graph Bonds:} If $|M_{ext}| = k$, the matching contains $k$ "long-range" bonds. These bonds do not interact with the Hamiltonian $H = \sum_{e \in E(G)} h_e$. Consequently, the diagonal energy $V(M)$ increases by at least $k$ units relative to a perfect graph matching.
\end{itemize}

\subsection{Formal Dirichlet Barrier Proof}
The Dirichlet eigenvalue $\lambda^{\text{Dir}}(B_r^c)$ is the ground state energy of the system restricted to configurations at distance $>r$ from $M_0$.

\begin{lemma}[Dirichlet Barrier]
For any state $f$ with $\text{supp}(f) \subseteq B_r(M_0)^c$:
\begin{equation}
    \langle f, \hat{H} f \rangle = \sum_{M \in \text{supp}(f)} V(M) |f(M)|^2 - \sum_{M, M'} A_{MM'} f(M)f(M')
\end{equation}
By the frustration-diffusion decomposition:
\begin{equation}
    \langle f, \hat{H} f \rangle = \sum_{M} (V(M) - \text{deg}_{\mathcal{G}}(M)) |f(M)|^2 + \mathcal{E}_{\mathcal{G}}(f)
\end{equation}
On a Ramanujan graph $G$, the following hold:
\begin{enumerate}
    \item \textbf{Frustration:} $V(M) - \text{deg}_{\mathcal{G}}(M) \ge \Delta_{gap} > 0$ because matchings far from $M_0$ necessarily involve more "non-graph" bonds or less-resonant configurations.
    \item \textbf{Expansion:} $\mathcal{E}_{\mathcal{G}}(f) \ge \nu \|f\|^2$ due to the Small-Set Expansion of the Dimer Move Graph.
\end{enumerate}
Thus, $\lambda^{\text{Dir}}(B_r^c) \ge \Delta_{gap} + \nu = \varepsilon$, a constant independent of $n$.
\end{lemma}

\subsection{Extensions: Unbalanced Partitions and Doped Systems}
While the current result focuses on balanced bipartite graphs ($|A| = |B|$), the framework provides a natural bridge to unbalanced systems. When $|B| = |A| + k$, the system is forced to contain $k$ unmatched vertices, or \textit{monomers}. 

\paragraph{The Doped Move Graph} In this regime, the configuration space $\Omega$ transitions from a Perfect Matching Complex to a Monomer-Dimer Complex. The fundamental "4-cycle swap" is augmented by \textit{alternating path moves}, where a monomer effectively performs a random walk by displacing adjacent dimers. 

\paragraph{Generalized Spectral Stability} The spectral gap $\nu$ remains robust in this extension due to the proof of the \textbf{Aldous Conjecture for the Interchange Process with Holes} \cite{caputo2010proof}. Caputo et al. established that the spectral gap of the many-body interchange process is equal to the single-particle gap of $G$, even when the number of particles is less than the number of vertices. This suggests that the \textbf{Dirichlet Barrier} (Pillar 3) remains $O(1)$ on Ramanujan expanders, preventing the localization of "holes" and maintaining the poly-time approximability of the doped ground state.

\paragraph{Future Directions} A significant challenge in the unbalanced case is the potential emergence of non-trivial Berry phases as monomers permute. Further work is required to establish whether the stoquasticity of the VB basis is preserved under these "hole-displacement" dynamics, or if the system maps to a more complex $Z_2$ gauge theory.

\begin{thebibliography}{9}
\bibitem{caputo2010proof}
P. Caputo, T. M. Liggett, and T. Richthammer. 
\textit{Proof of Aldous' spectral gap conjecture.} 
Journal of the American Mathematical Society, 23(3), 831-851 (2010).
\end{thebibliography}


\begin{thebibliography}{9}
\bibitem{AL20} Alev, V., \& Lau, L. C. (2020). ``Improved Analysis of Markov Chains for Monte Carlo Integration and Sampling.'' \textit{STOC}.
\bibitem{Opp18} Oppenheim, I. (2018). ``Local-to-global spectral expansion of simplicial complexes.'' \textit{Inventiones mathematicae}.
\bibitem{JSV} Jerrum, M., Sinclair, A., \& Vigoda, E. (2004). ``A polynomial-time approximation algorithm for the determinant of a matrix with non-negative entries.'' \textit{J. ACM}.
\end{thebibliography}

\end{document}