\documentclass{article}
\usepackage[utf8]{inputenc}
\usepackage{amsmath, amssymb, amsthm, hyperref}

\newtheorem{theorem}{Theorem}
\newtheorem{lemma}{Lemma}
\newtheorem{definition}{Definition}
\newtheorem{proposition}{Proposition}

\title{Notes: Polynomial Complexity of Quantum Max-Cut on Expanders}
\author{Technical Summary for Whiteboard Presentation}
\date{2026}

\begin{document}

\maketitle

\section{High-Level Motivation: The "Self-Guiding" Ground State}

The central challenge in stoquastic complexity (Bravyi-Terhal, 2009) is the "mixing" and "bottleneck" problem: even if a Hamiltonian is sign-problem free, its ground state may be spread across an exponential configuration space.

\subsection{The Core Intuition}
We argue that for \textbf{Quantum Max-Cut (QMC)} on \textbf{Ramanujan Expanders}, the ground state is not a diffuse cloud, but a \textbf{local soliton} in the space of perfect matchings. 
\begin{itemize}
    \item \textbf{Aldous Conjecture Connection:} On expanders, many-body "interchange" processes (dimer swaps) inherit the spectral gap of the single-edge swaps. This prevents bottlenecks.
    \item \textbf{Bravyi-Terhal Integration:} We utilize the stoquastic mapping to treat the ground state as the equilibrium of a local random walk, but we prove this walk is "trapped" in a potential well.
\end{itemize}

\subsection{The Roadmap}
\begin{enumerate}
    \item \textbf{Map:} Transform the spin Hilbert space to the \textbf{Perfect Matching Move Graph} $\mathcal{G}$.
    \item \textbf{Anchor:} Establish a \textbf{Diagonal Energy Floor} that penalizes non-graph matchings.
    \item \textbf{Confine:} Use \textbf{High-Dimensional Expansion (HDX)} to prove a constant Dirichlet barrier.
    \item \textbf{Truncate:} Show the ground state weight decays exponentially, allowing for \textbf{Constant Radius Truncation}.
\end{enumerate}

\hrule

\section{The Technical Deep Dive}

\subsection{Object 1: The Orthonormalized Move Graph}
Let $G=(V,E)$ be a $d$-regular bipartite Ramanujan graph. We define the Hilbert space $\mathcal{H}_{S=0}$ via the Valence Bond (VB) basis of perfect matchings $\Omega$. 

\begin{definition}[Dimer Move Graph $\mathcal{G}$]
The move graph $\mathcal{G}$ has vertices $M \in \Omega$. An edge exists between $M$ and $M'$ if they differ by a \textbf{4-cycle swap}: $M \Delta M' = C_4$, where $C_4 \subset G$.
\end{definition}

In the orthonormalized basis $\{|e_M\rangle\}$, the Hamiltonian admits the stoquastic decomposition $\hat{H} = \mathbf{V} - \mathbf{A}$, where $\mathbf{A}$ is the adjacency matrix of $\mathcal{G}$.

\subsection{Pillar I: The Diagonal Gap (The Potential Well)}
To prevent the state from diffusing, we need the "classical" energy of any matching to be significantly higher than the quantum ground state $E_0$.

\begin{lemma}[Diagonal Floor]
For a $d$-regular Ramanujan graph, let $V(M) = \langle e_M | \hat{H} | e_M \rangle$. There exists a constant $\gamma \approx \frac{1}{d}$ such that for any basis matching $M$:
\begin{equation}
    \frac{1}{n} (V(M) - E_0) \ge \gamma
\end{equation}
\end{lemma}
\textit{Proof Sketch:} $E_0$ is lowered relative to the average diagonal energy by the "resonance" of the 4-cycle swaps. In an expander, the move graph degree is $\Omega(nd)$, ensuring $E_0 \le \bar{V} - \text{const} \cdot n$.

\subsection{Pillar II: HDX and the Dirichlet Barrier (The Kinetic Friction)}
We must guarantee that the "quantum pressure" pushes the state back toward the seed matching $M_0$.

\begin{lemma}[Constant Dirichlet Barrier]
Let $\mathcal{H}_r$ be the ball of radius $r$ around seed $M_0$ in $\mathcal{G}$. The Dirichlet eigenvalue of the exterior $\mathcal{H}_r^c$ satisfies:
\begin{equation}
    \lambda^{\text{Dir}}(\mathcal{H}_r^c) \ge E_0 + \nu
\end{equation}
where $\nu > 0$ is a constant independent of $n$.
\end{lemma}
\textit{Proof Sketch:} By the \textbf{Aldous Conjecture} (Caputo et al., 2009), the spectral gap of the interchange process on $G$ is equal to the gap of $G$. Since $G$ is Ramanujan, the local links of the matching complex are expanders. By Oppenheim's Local-to-Global theorem, small sets in $\mathcal{G}$ expand at a constant rate $\nu$.

\hrule

\section{The Final Takeaway: Constant Radius Truncation}

\subsection{The Localization Theorem}
The combined constant barrier $\Delta = \gamma + \nu$ acts as a "mass term" in the move-graph propagator. The ground state $|\psi_0\rangle$ satisfies an Agmon-type exponential decay:
\begin{equation}
    |\psi_0(M)| \le C \exp(-dist(M, M_0) \cdot \sqrt{\Delta})
\end{equation}

\subsection{Algorithmic Complexity}
To approximate $|\psi_0\rangle$ to precision $\delta$:
\begin{enumerate}
    \item \textbf{Truncation Radius:} We choose $r$ such that $e^{-r\sqrt{\Delta}} \le \delta$. Crucially, \textbf{$r$ is a constant} independent of $n$.
    \item \textbf{Subspace Dimension:} $dim(\mathcal{H}_r) \le (nd)^r = O(n^r)$.
    \item \textbf{Solution:} We perform Imaginary Time Evolution (ITE) $e^{-\beta \hat{H}_{proj}}$ within $\mathcal{H}_r$. 
\end{enumerate}

\textbf{Final Conclusion:} 
Since $r$ is constant, the subspace $\mathcal{H}_r$ is polynomial-sized. The "self-guiding" nature of the expander geometry ensures the ground state is captured within this local neighborhood. The problem of finding the ground state of QMC on bipartite Ramanujan expanders is therefore in \textbf{P}.



\pagebreak 




\end{document}