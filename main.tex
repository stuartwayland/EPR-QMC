% Max-Cut Hamiltonian notes
\documentclass[11pt]{article}
\usepackage[margin=1in]{geometry}
\usepackage{amsmath,amssymb,amsfonts}
\usepackage{braket}

% Dirac notation shortcuts
\newcommand{\ip}[2]{\langle #1 \mid #2 \rangle}


\begin{document}

In the Max-Cut Hamiltonian, each local operator is defined by the edge Hamiltonian
\[
	h_{ab} = \tfrac{1}{2}\bigl(\mathbb{I} - \mathrm{SWAP}_{ab}\bigr)\otimes \mathbb{I}_{[n]\setminus\{a,b\}}.
\]

Therefore, the total Hamiltonian for a graph $G$ is
\[
	H_G = \sum_{e\in E} w_e\, h_e.
\]

Moreover, the state
\[
	|\psi\rangle = \tfrac{1}{\sqrt{2}}\bigl(|01\rangle - |10\rangle\bigr)
\]
achieves the maximum energy of $h_{ab}$.

Recent work has found success in considering the matching state
\[
	|\rho_m\rangle = |\psi\rangle_{ab} \otimes |\psi\rangle_{cd} \otimes \cdots \otimes |0000\rangle,
\]
corresponding to a maximal matching $m = \{(a,b), (c,d), \ldots\}$.

\paragraph{Expansion in the computational basis.}
For the singlet on qubits $a,b$,
\[
\ket{\psi^-}_{ab}
=\tfrac{1}{\sqrt{2}}(\ket{01}-\ket{10})
=\tfrac{1}{\sqrt{2}}\sum_{i,j\in\{0,1\}}\varepsilon_{ij}\,\ket{i}_a\ket{j}_b,
\quad
\varepsilon=\begin{pmatrix}0&1\\-1&0\end{pmatrix}.
\]
Hence for a perfect matching $m$ on $[n]$,
\[
\ket{\Phi_m}
=\bigotimes_{(a,b)\in m}\ket{\psi^-}_{ab}
=\left(\tfrac{1}{\sqrt{2}}\right)^{n/2}
\sum_{x\in\{0,1\}^n}
\Biggl[\prod_{(a,b)\in m}\varepsilon_{x_a x_b}\Biggr]\ket{x}.
\]
Each basis vector $\ket{x}$ receives a coefficient given by the product of
$\varepsilon_{x_a x_b}$ over the disjoint pairs in $m$.

\paragraph{Inner product.}
Similarly,
\[
\ket{\Phi_{m'}}
=\left(\tfrac{1}{\sqrt{2}}\right)^{n/2}
\sum_{y\in\{0,1\}^n}
\Biggl[\prod_{(c,d)\in m'}\varepsilon_{y_c y_d}\Biggr]\ket{y}.
\]
Then
\[
\ip{\Phi_m}{\Phi_{m'}}
=\left(\tfrac{1}{\sqrt{2}}\right)^{n}
\sum_{x,y\in\{0,1\}^n}
\Biggl[\prod_{(a,b)\in m}\varepsilon_{x_a x_b}\Biggr]
\Biggl[\prod_{(c,d)\in m'}\varepsilon_{y_c y_d}\Biggr]
\ip{x}{y}.
\]
Using orthonormality $\ip{x}{y}=\delta_{xy}$, the double sum collapses:
\[
\ip{\Phi_m}{\Phi_{m'}}
=2^{-n/2}
\sum_{x\in\{0,1\}^n}
\Biggl[\prod_{(a,b)\in m}\varepsilon_{x_a x_b}\Biggr]
\Biggl[\prod_{(c,d)\in m'}\varepsilon_{x_c x_d}\Biggr].
	ag{1}\label{eq:overlap-compact}
\]
This is the desired compact form for the overlap of two perfect-matching (valence-bond) states.

\section*{From edge products to cycle products}

\paragraph{Cycle decomposition.}
Let $m,m'$ be perfect matchings on $[n]$.
In the multigraph $m\cup m'$, every vertex has degree $2$ (one edge from $m$, one from $m'$),
so $m\cup m'$ is a disjoint union of even cycles
\[
C_1,\dots,C_\ell,\qquad |C_i|=2r_i,\ \ r_i\ge 1,\ \ \sum_{i=1}^{\ell} r_i=\frac{n}{2}.
\]
Along each $C_i$ the edges alternate between $m$ and $m'$.

\paragraph{Rewriting the edge product by cycles.}
Starting from \eqref{eq:overlap-compact},
\[
\prod_{(a,b)\in m}\varepsilon_{x_a x_b}\;\prod_{(c,d)\in m'}\varepsilon_{x_c x_d},
\]
note that each edge of $m$ and $m'$ lies in exactly one cycle $C_i$, and distinct cycles share no vertices.
Hence the product factors as
\[
\prod_{(a,b)\in m}\varepsilon_{x_a x_b}\;\prod_{(c,d)\in m'}\varepsilon_{x_c x_d}
\;=\;
\prod_{i=1}^{\ell}\ \mathcal{F}(C_i;x),
\]
where for a fixed ordering $C_i=(v^{(i)}_1,\dots,v^{(i)}_{2r_i})$ with alternating $m/m'$ edges we define
\[
\mathcal{F}(C_i;x)
\;:=\;
\varepsilon_{x_{v^{(i)}_1}x_{v^{(i)}_2}}
\varepsilon_{x_{v^{(i)}_2}x_{v^{(i)}_3}}
\cdots
\varepsilon_{x_{v^{(i)}_{2r_i}}x_{v^{(i)}_1}}.
\]

\paragraph{Factorization of the sum.}
Let $X_i=\{x_v:\ v\in C_i\}$.
Since cycles are vertex-disjoint, $\{x_1,\dots,x_n\}=X_1\;\dot\cup\;\cdots\;\dot\cup\;X_\ell$, and each
$\mathcal{F}(C_i;x)$ depends only on $X_i$.
Therefore the sum in \eqref{eq:overlap-compact} splits:
\[
\sum_{x\in\{0,1\}^n}\ \prod_{i=1}^{\ell}\mathcal{F}(C_i;x)
\;=\;
\prod_{i=1}^{\ell}\ \Biggl(\sum_{X_i\in\{0,1\}^{C_i}} \mathcal{F}(C_i;X_i)\Biggr),
\]
which is the desired reduction from edge-wise products to independent cycle-wise contributions.

\section*{Evaluation of a single cycle}
\paragraph{Setup.}
Fix one cycle $C=(v_1,v_2,\dots,v_{2r})$ of $m\cup m'$,
with edges alternating between those from $m$ and $m'$.
The contribution from this cycle to the overlap is
\[
S(C)
=\sum_{x_{v_1},\dots,x_{v_{2r}}\in\{0,1\}}
\varepsilon_{x_{v_1}x_{v_2}}\,
\varepsilon_{x_{v_2}x_{v_3}}\,
\varepsilon_{x_{v_3}x_{v_4}}\cdots
\varepsilon_{x_{v_{2r}}x_{v_1}}.
\]

\paragraph{Entries of the antisymmetric matrix.}
Here $\varepsilon_{ij}$ denotes the \emph{scalar entry} of the fixed $2\times2$
antisymmetric matrix
\[
\varepsilon=
\begin{pmatrix}
0 & 1\\[2pt]
-1 & 0
\end{pmatrix},
\qquad
\text{so that}\quad
\varepsilon_{00}=0,\;
\varepsilon_{01}=1,\;
\varepsilon_{10}=-1,\;
\varepsilon_{11}=0.
\]
Each $\varepsilon_{x_u x_v}$ in the product is simply one of these numerical
entries, evaluated at the particular values of $x_u,x_v\in\{0,1\}$.

\paragraph{Key identity.}
Because $\varepsilon$ is antisymmetric and satisfies
$\varepsilon\,\varepsilon^{\mathsf T}=I_2$, its components obey
\[
\sum_{x\in\{0,1\}} \varepsilon_{a x}\,\varepsilon_{x b} = \delta_{ab}.
\tag{$\ast$}
\]
This identity is purely numeric: it states that when the middle index $x$ is
summed over, the product $\varepsilon_{a x}\varepsilon_{x b}$ is nonzero only
when $a=b$.

\paragraph{Iterative elimination.}
In the cycle sum $S(C)$, successive factors share exactly one index.
Using $(\ast)$ to sum over those shared variables collapses pairs of
$\varepsilon$'s into Kronecker deltas:
\[
\sum_{x_{v_2}}\varepsilon_{x_{v_1}x_{v_2}}\varepsilon_{x_{v_2}x_{v_3}}
=\delta_{x_{v_1}x_{v_3}},
\]
and repeating this contraction around the cycle enforces equality of all
remaining indices. After all internal variables are eliminated, only one free
binary variable (say $x_{v_1}\in\{0,1\}$) remains, giving
\[
S(C)=\sum_{x_{v_1}\in\{0,1\}}1=2.
\]

\paragraph{Result.}
Thus each closed cycle in $m\cup m'$ contributes a scalar value $S(C)=2$ to the
overlap, independent of its length. The factor $2$ arises from the two possible
assignments of the common spin shared by all vertices in the cycle.


\paragraph{Final boxed result.}
Combining the cycle factorization with $S(C)=2$ for each of the $\ell$ disjoint cycles in $m\cup m'$, we obtain
\[
\boxed{\ \ip{\Phi_m}{\Phi_{m'}}\;=\;2^{\,\ell}\,2^{-n/2}\;=\;2^{\,\ell - n/2}\ }\,,
\]
where $\ell$ is the number of cycles in the disjoint union $m\cup m'$.

\pagebreak 

\section*{Extending to maximal (not necessarily perfect) matchings}

\paragraph{State convention.}
For a (maximal) matching $m=\{(a_1,b_1),\dots,(a_k,b_k)\}$ on $[n]$,
let $V_m=\{a_1,b_1,\dots,a_k,b_k\}$ and define
\[
\ket{\rho_m}
=\Bigl(\bigotimes_{(a,b)\in m}\ket{\psi^-}_{ab}\Bigr)
\otimes \Bigl(\bigotimes_{u\in [n]\setminus V_m}\ket{0}_u\Bigr),
\qquad
\ket{\psi^-}=\tfrac{1}{\sqrt{2}}(\ket{01}-\ket{10}).
\]

\paragraph{Zero-overlap criterion.}
If $V_m\neq V_{m'}$ then $\braket{\rho_m|\rho_{m'}}=0$.
Indeed, pick $u\in V_m\triangle V_{m'}$. WLOG $u\in V_m$ and $u\notin V_{m'}$.
Then in the inner product one factor is $\bra{\psi^-}_{uv}$ (for the unique $v$ paired with $u$ in $m$)
and the other gives $\ket{0}_u\otimes(\cdots)_v$, so
\[
\bra{0}_u\bra{\cdots}_v\ \ket{\psi^-}_{uv}
=\tfrac{1}{\sqrt{2}}\bra{\cdots}_v\ket{1}_v=0,
\]
since $\ket{\cdots}_v=\ket{0}_v$ by our convention on unmatched sites.

\paragraph{Reduction to the covered subset.}
Assume $V_m=V_{m'}=:S$ with $|S|=2k$.
On $S$ both $m$ and $m'$ are \emph{perfect} matchings (of the induced set $S$), and the
outside qubits all contribute $\prod_{u\notin S}\braket{0|0}=1$.
Hence
\[
\braket{\rho_m|\rho_{m'}}=\braket{\Phi_m|\Phi_{m'}},
\qquad
\ket{\Phi_m}=\bigotimes_{(a,b)\in m}\ket{\psi^-}_{ab}
\ \text{ on the register } S.
\]

\paragraph{Cycle decomposition on $S$.}
Superimpose $m$ and $m'$ on $S$. Each vertex has degree $2$ (one edge from each matching),
so $m\cup m'$ decomposes into $\ell=\ell(m,m')$ disjoint even cycles
$C_1,\dots,C_\ell$ with lengths $|C_i|=2r_i$ and $\sum_i r_i=k$.

\paragraph{Main formula.}
Exactly as in the perfect-matching case, each cycle evaluates to $2$, so
\[
\boxed{\ \braket{\rho_m|\rho_{m'}}=
\begin{cases}
2^{\,\ell(m,m')-k}, & \text{if } V_m=V_{m'},\ k=|m|=|m'|,\\[3pt]
0, & \text{otherwise.}
\end{cases}\ }
\]

\paragraph{Remarks.}
(i) If $|m|\neq |m'|$ then $|V_m|\neq |V_{m'}|$, hence the overlap is $0$.
(ii) The exponent $k$ is the number of singlet pairs (i.e.\ half the size of $S$).
(iii) If a different filler product state is used on unmatched sites, replace the zero-overlap
argument accordingly; for the choice $\ket{0}^{\otimes (n-2k)}$ above, the criterion is sharp.




\end{document}

