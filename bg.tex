\documentclass[12pt]{article}
\usepackage{amsmath, amssymb, amsthm, bm}

\title{Whiteboard Talk Outline: Classical Simulation of Quantum Ferromagnets}
\date{}
\begin{document}
\maketitle

\section{Overview and Goal}

We present the core ideas of Bravyi--Gosset (2016), who show that a broad class
of quantum ferromagnetic Hamiltonians have partition functions that can be
approximated in classical polynomial time.

\bigskip

\textbf{Main pipeline:}
\[
H \quad\Longrightarrow\quad e^{-\beta H}
 \approx \prod_{t=1}^J G_t
 \quad\Longrightarrow\quad \text{gadgets for each $G_t$}
 \quad\Longrightarrow\quad \Gamma\ (\text{matching graph})
\]
\[
Z(\beta,H)=\mathrm{Tr}(e^{-\beta H})\approx \mathrm{PerfMatch}(\Gamma).
\]

The Jerrum--Sinclair algorithm then gives a randomized PTAS for counting perfect
matchings of $\Gamma$, implying an efficient classical simulation of $H$.

%%%%%%%%%%%%%%%%%%%%%%%%%%%%%%%%%%%%%%%%%%%%%%%%%%%%%%%%%%%%%%%%%%%%%%%%%%%%%%%
\section{The Hamiltonian}

The paper studies Hamiltonians of the form
\begin{equation}
H = \sum_{i<j} \left( -b_{ij} X_i X_j + c_{ij} Y_i Y_j \right)
      + \sum_i d_i (I + Z_i),
\label{eq:H}
\end{equation}
subject to the \textbf{ferromagnetic constraint}
\[
|c_{ij}| \le b_{ij}, \qquad b_{ij},|c_{ij}|,|d_i| \le 1.
\]

Physically this family includes:
\begin{itemize}
\item Ferromagnetic XY model ($c_{ij} = -b_{ij}$),
\item Transverse-field Ising model ($c_{ij}=0$),
\item Interpolations between them.
\end{itemize}

Under the constraint $|c_{ij}| \le b_{ij}$, all \emph{off-diagonal} matrix
elements of $H$ in the computational basis are \textbf{non-positive}:
\[
\langle x | H | y \rangle \le 0 \quad \text{for } x\neq y.
\]
Hence $H$ is \textbf{stoquastic}. This will ensure that imaginary-time evolution
$e^{-\beta H}$ has non-negative matrix elements, avoiding the sign problem.

%%%%%%%%%%%%%%%%%%%%%%%%%%%%%%%%%%%%%%%%%%%%%%%%%%%%%%%%%%%%%%%%%%%%%%%%%%%%%%%
\section{Imaginary Time Evolution}

We wish to approximate the partition function
\[
Z(\beta,H)=\mathrm{Tr}(e^{-\beta H}).
\]

Imaginary-time evolution replaces the real-time propagator $e^{-iHt}$ with
\[
e^{-\beta H},
\]
which behaves like a ``Boltzmann weight'' operator that filters toward low
energy states. For $\beta$ large, it approximates the ground-state projector.

\subsection{Why imaginary time avoids the sign problem}

Because $H$ is stoquastic, $-\beta H$ has \emph{non-negative} off-diagonal
entries. Thus
\[
e^{-\beta H} = \sum_{k=0}^\infty \frac{(-\beta H)^k}{k!}
\]
is a sum of matrices with non-negative entries. Hence:
\[
\langle x | e^{-\beta H} | y \rangle \ge 0
\]
for all basis states $x,y$.

This non-negativity is crucial: it permits a representation of the matrix
elements of $e^{-\beta H}$ as sums of weights in a classical counting problem
(without cancellations). This is exactly what allows mapping to perfect
matchings.

%%%%%%%%%%%%%%%%%%%%%%%%%%%%%%%%%%%%%%%%%%%%%%%%%%%%%%%%%%%%%%%%%%%%%%%%%%%%%%%
\section{Decomposing $e^{-\beta H}$ via Trotter--Suzuki}

Write the Hamiltonian as a sum of local terms:
\[
H = \sum_\alpha H_\alpha,
\]
where each $H_\alpha$ acts on at most two qubits.

For sufficiently large $J$,
\[
e^{-\beta H}
\approx
\left(
\prod_\alpha e^{-(\beta/J) H_\alpha}
\right)^J
\quad +\quad O(\beta^2/J).
\]

Define the sequence of gates
\[
G_1,G_2,\dots,G_J
\]
where each $G_t$ is either a one-qubit gate $e^{-(\beta/J) d_i (I+Z_i)}$ or a
two-qubit gate $e^{-(\beta/J)(-b_{ij} X_i X_j + c_{ij} Y_i Y_j)}$.

Key fact:
\[
\text{All }G_t\text{ have non-negative matrix elements.}
\]

Thus the circuit operator
\[
G_J G_{J-1}\cdots G_1
\]
has non-negative matrix entries and approximates $e^{-\beta H}$.

%%%%%%%%%%%%%%%%%%%%%%%%%%%%%%%%%%%%%%%%%%%%%%%%%%%%%%%%%%%%%%%%%%%%%%%%%%%%%%%
\section{Mapping Each Gate to a Small Gadget}

For each local gate $G_t$, construct a \textbf{classical weighted graph gadget}
$F_t$ with designated \emph{input} and \emph{output} vertices.

The gadget encodes the following property:

For computational basis states $x,y \in \{0,1\}^k$ (where $k=1$ or $2$),
let $F_t(x,y)$ be the graph obtained from the gadget by:
\begin{itemize}
\item deleting the input vertex $i$ if $x_i=1$,
\item deleting the output vertex $i$ if $y_i=1$.
\end{itemize}

Then the sum over perfect matchings of $F_t(x,y)$ satisfies
\begin{equation}
\mathrm{PerfMatch}(F_t(x,y)) = \langle y | G_t | x \rangle.
\label{eq:gadget-property}
\end{equation}

Thus the local matrix elements of $G_t$ are realized as weighted matching counts
over a \emph{constant-size} graph.

\subsection{Why this works}

Because $G_t$ is a positive-semidefinite matrix with non-negative entries,
every $2\times 2$ or $4\times 4$ block decomposition can be represented as the
partition function of a small planar graph. The paper shows explicit gadgets
for:
\begin{itemize}
\item one-qubit gates $f(t)$,
\item two-qubit diagonal-ish gates $g(t)$,
\item two-qubit off-diagonal gates $h(t)$.
\end{itemize}

The edge weights of the gadget encode the four (or sixteen) entries of $G_t$.
Stoquasticity guarantees positive weights.

%%%%%%%%%%%%%%%%%%%%%%%%%%%%%%%%%%%%%%%%%%%%%%%%%%%%%%%%%%%%%%%%%%%%%%%%%%%%%%%
\section{Composing Gadgets into the Full Graph $\Gamma$}

Now consider the ordered product
\[
U = G_J G_{J-1}\cdots G_1.
\]

For each gate $G_t$ with gadget $F_t$, join the gadgets as follows:

\begin{itemize}
\item Identify the output vertices of $F_t$ for qubit $i$ with the input
  vertices of $F_{t+1}$ for the same qubit.
\item Use unit-weight edges for these identifications.
\item For $\mathrm{Tr}(U)$, glue the final outputs of $F_J$ back to the inputs
  of $F_1$ (cyclic boundary condition).
\end{itemize}

The resulting graph $\Gamma$ contains $O(Jn)$ vertices and $O(Jn)$ edges.

\subsection{Key identity}

Because of property \eqref{eq:gadget-property} and because taking the trace
corresponds to summing over all bitstrings $x = y$,
\begin{equation}
\mathrm{PerfMatch}(\Gamma)
=
\mathrm{Tr}(U)
\approx
\mathrm{Tr}(e^{-\beta H})
=
Z(\beta,H).
\end{equation}

Thus the quantum partition function is equal (up to Trotter error) to the
weighted perfect-matching partition function of the classical graph $\Gamma$.

%%%%%%%%%%%%%%%%%%%%%%%%%%%%%%%%%%%%%%%%%%%%%%%%%%%%%%%%%%%%%%%%%%%%%%%%%%%%%%%
\section{Why This Avoids the Sign Problem}

The sign problem arises in Feynman path-integral or worldline Monte Carlo when
weights of paths alternate in sign, causing exponential cancellation.

Here:
\[
\langle y | G_t | x \rangle \ge 0
\quad\Rightarrow\quad
\text{All edge weights in the gadgets are non-negative}.
\]
Therefore each perfect matching carries a \emph{positive} weight, and the global
sum has no cancellations.

This is precisely why stoquasticity is a necessary condition for the reduction.

%%%%%%%%%%%%%%%%%%%%%%%%%%%%%%%%%%%%%%%%%%%%%%%%%%%%%%%%%%%%%%%%%%%%%%%%%%%%%%%
\section{Classical Approximation of $\mathrm{PerfMatch}(\Gamma)$}

Although computing the exact number of perfect matchings is $\#\mathrm{P}$-hard
in general, Jerrum--Sinclair (and later Jerrum--Vazirani) give a polynomial-time
randomized approximation scheme for \emph{non-negative weighted} perfect
matchings on general graphs.

The graph $\Gamma$ satisfies:
\[
|\Gamma| = \mathrm{poly}(n,\beta,\varepsilon^{-1}),
\]
so the randomized PTAS gives an approximation
\[
\widehat{Z} \approx Z(\beta,H)
\]
in polynomial time.

%%%%%%%%%%%%%%%%%%%%%%%%%%%%%%%%%%%%%%%%%%%%%%%%%%%%%%%%%%%%%%%%%%%%%%%%%%%%%%%
\section{Conclusion}

We have reduced the quantum many-body problem
\[
Z(\beta,H)=\mathrm{Tr}(e^{-\beta H})
\]
for the ferromagnetic Hamiltonian \eqref{eq:H} to a classical matching-counting
problem. The key enabling features were:
\begin{itemize}
\item stoquasticity $\Rightarrow$ positivity of imaginary-time matrix elements,
\item Trotter decomposition into local positive gates,
\item explicit gadget representations for each gate,
\item polynomial-size global gadget graph.
\end{itemize}

This produces a rare example of a nontrivial quantum spin model that is
classically simulable (approximately) in polynomial time.

\end{document}