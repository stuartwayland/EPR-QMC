\documentclass[11pt]{article}
\usepackage{amsmath,amssymb,amsthm,bm}
\usepackage[margin=1in]{geometry}

\newcommand{\ket}[1]{\left|#1\right\rangle}
\newcommand{\bra}[1]{\left\langle#1\right|}
\newcommand{\braket}[2]{\left\langle #1 \middle| #2 \right\rangle}

\begin{document}

\begin{center}
\Large QMC on Even Cycles
\end{center}

\section*{1. QMC Hamiltonian on an even cycle and mapping to XXX}

Let $C_{2L}$ be the even cycle with sites $i=1,\dots,2L$ and periodic boundary conditions $i\equiv i+2L$.
We start with the benchmark \emph{Quantum Max-Cut} (QMC) Hamiltonian (on a graph $G$): :
\begin{equation}
  H_{\mathrm{QMC}}
  \;=\;
  \sum_{(i,j)\in E(G)} w_{ij}\,
  \frac{1}{4}\!\left(
    I - X_i X_j - Y_i Y_j - Z_i Z_j
  \right),
  \label{eq:QMC}
\end{equation}
where $X_i,Y_i,Z_i$ are the Pauli operators on qubit $i$.
For our case, the graph is the even cycle $G=C_{2L}$ with uniform edge weights $w_{ij}=J>0$, so
\begin{equation}
  H_{\mathrm{QMC}}
  =
  \frac{J}{4}\sum_{i=1}^{2L}
  \!\left(
    I - X_i X_{i+1} - Y_i Y_{i+1} - Z_i Z_{i+1}
  \right).
  \label{eq:QMC_cycle}
\end{equation}
The constant term $\frac{J}{4}\sum_i I$ merely shifts the energy by $J L$; the nontrivial dynamics arise from
\begin{equation}
  \widetilde H_{\mathrm{QMC}}
  =
  -\frac{J}{4}\sum_{i=1}^{2L}
  \!\left(
    X_i X_{i+1} + Y_i Y_{i+1} + Z_i Z_{i+1}
  \right).
  \label{eq:H_tilde}
\end{equation}

\smallskip
\noindent
\textbf{Mapping to the Heisenberg (XXX) chain.}
The even cycle $C_{2L}$ is bipartite, with two equal sublattices
$A=\{1,3,5,\dots,2L-1\}$ and $B=\{2,4,6,\dots,2L\}$.
Define a local unitary rotation on all even (i.e.\ $B$-sublattice) sites:
\begin{equation}
  U_i \;=\; e^{-i\frac{\pi}{2}\,Y_i},
  \qquad i\in B,
  \label{eq:rotation_def}
\end{equation}
so that
\begin{equation}
  U_i^\dagger X_i U_i = Z_i,
  \quad
  U_i^\dagger Z_i U_i = -X_i,
  \quad
  U_i^\dagger Y_i U_i = Y_i.
  \label{eq:rotation_action}
\end{equation}
Let $U=\prod_{i\in B} U_i$ denote the global unitary acting on all even sites.
Conjugating $\widetilde H_{\mathrm{QMC}}$ by $U$ gives
\begin{align}
  U^\dagger \widetilde H_{\mathrm{QMC}} U
  &=
  -\frac{J}{4}
  \sum_{i=1}^{2L}
  \!\left(
    U^\dagger X_i X_{i+1} U
    + U^\dagger Y_i Y_{i+1} U
    + U^\dagger Z_i Z_{i+1} U
  \right) \nonumber\\[3pt]
  &=
  +\frac{J}{4}\sum_{i=1}^{2L}
  \!\left(
    X_i X_{i+1} + Y_i Y_{i+1} + Z_i Z_{i+1}
  \right)
  \;=\;
  J\sum_{i=1}^{2L} \bm S_i\!\cdot\!\bm S_{i+1},
  \label{eq:H_XXX}
\end{align}
where $\bm S_i = \tfrac12(X_i,Y_i,Z_i)$.
Thus, up to a constant energy shift and the overall scale $J$, the Quantum Max-Cut Hamiltonian is \emph{unitarily equivalent} to the
antiferromagnetic Heisenberg (XXX) model:
\begin{equation}
  H_{\mathrm{XXX}}
  \;=\;
  J\sum_{i=1}^{2L} \bm S_i\!\cdot\!\bm S_{i+1},
  \qquad J>0.
  \label{eq:XXX}
\end{equation}
Because the mapping is a local SU(2) rotation, all spectral and correlation properties are preserved:
$H_{\mathrm{QMC}}$ and $H_{\mathrm{XXX}}$ have identical eigenvalues and entanglement structure.

\smallskip
\noindent
\textbf{Ground-state structure.}
Since $C_{2L}$ is bipartite with $|A|=|B|=L$, the model is unfrustrated and SU(2)-invariant.
By the Lieb--Mattis theorem on bipartite antiferromagnets,
the ground state lies in the total-spin singlet sector $S_{\rm tot}=0$.

\smallskip
\noindent
\textbf{Remark on QMC/EPR variants.}
If one starts from a ``quantum Max-Cut'' variant involving only $\sigma_i^x\sigma_j^x$
couplings (as in transverse-field or EPR-type formulations),
an analogous sublattice rotation brings it to the isotropic XXX form~\eqref{eq:XXX}
up to constants whenever the underlying graph is bipartite.
Hence, for the even cycle, we adopt~\eqref{eq:XXX} as the canonical form.

\section*{2. Bethe ansatz solution (periodic, even length, $S^z_{\rm tot}=0$)}
Work in the computational basis with $M=L$ flipped spins:
\begin{equation}
  \ket{\Psi}
  = \sum_{1\le x_1<\cdots<x_M\le 2L}
  \psi(x_1,\dots,x_M)\, S^-_{x_1}\cdots S^-_{x_M}\ket{\!\uparrow^{\otimes 2L}\!}.
\end{equation}
The \emph{coordinate Bethe ansatz} amplitude is
\begin{equation}
  \psi(x_1,\dots,x_M)=
  \sum_{P\in S_M} A(P)\,
  \exp\!\Big(i\sum_{j=1}^{M} k_{P_j}\, x_j\Big),
  \qquad
  A(P)=\prod_{1\le j<\ell\le M}
  \frac{\sin\frac{k_{P_j}-k_{P_\ell}}{2}-i}{\sin\frac{k_{P_j}-k_{P_\ell}}{2}+i},
\end{equation}
with quasi-momenta $\{k_j\}_{j=1}^{M}$ solving the Bethe equations
\begin{equation}
  e^{i k_j (2L)}=\prod_{\ell\neq j}
  \frac{\sin\frac{k_j-k_\ell}{2}+i}{\sin\frac{k_j-k_\ell}{2}-i},
  \qquad j=1,\dots,M.
\end{equation}
For the ground state, the $k_j$ are real and symmetrically fill a Fermi sea. The ground-state energy is
\begin{equation}
  E_0
  = \sum_{j=1}^{M}\,\big( \tfrac{J}{2}(\cos k_j -1)\big),
\end{equation}
and the state is a unique translationally invariant singlet ($S_{\rm tot}=0$).

\section*{3. Valence-bond (VB) basis for the singlet sector}
Let $\mathcal{M}$ be the set of perfect matchings $M$ on $\{1,\dots,2L\}$. For $(i,j)\in M$ define the singlet on sites $i,j$:
\(
\ket{\psi^-_{ij}}=\tfrac{1}{\sqrt{2}}\big(\ket{\uparrow_i\downarrow_j}-\ket{\downarrow_i\uparrow_j}\big).
\)
A VB basis vector is the tensor product
\begin{equation}
  \ket{\mathrm{VB}(M)} \;=\; \bigotimes_{(i,j)\in M}\ket{\psi^-_{ij}}.
\end{equation}
The set $\{\ket{\mathrm{VB}(M)}: M\in\mathcal{M}\}$ spans the full $S_{\rm tot}=0$ subspace (overcomplete). Since the Bethe ground state is a singlet, it admits an exact expansion
\begin{equation}
  \ket{\Psi_0}
  \;=\;
  \sum_{M\in\mathcal{M}} \psi(M)\,\ket{\mathrm{VB}(M)}.
  \label{eq:VBexpansion}
\end{equation}

\section*{Change of basis: Bethe (computational) $\to$ Valence--Bond (perfect matchings)}

\subsection*{Valence--bond basis on the bipartite even cycle}
Let $C_{2L}$ be bipartitioned as $A=\{1,3,\dots,2L-1\}$ and $B=\{2,4,\dots,2L\}$.
For any perfect matching $M$ pairing each $a\in A$ to a unique $b\in B$, define the oriented
two--site singlet (with fixed orientation $A\to B$)
\begin{equation}
  \ket{s_{ab}}
  \;=\;\frac{1}{\sqrt{2}}\Big(\ket{\uparrow_a\downarrow_b}-\ket{\downarrow_a\uparrow_b}\Big),
  \qquad (a,b)\in M,
\end{equation}
and the (non--normalized) valence--bond (VB) state
\begin{equation}
  \ket{M} \;=\; \bigotimes_{(a,b)\in M} \ket{s_{ab}}.
\end{equation}
The set $\{\ket{M}\}$ spans the total--singlet subspace but is non--orthogonal and overcomplete.
With the above $A\to B$ orientation, Marshall's sign rule ensures that ground--state
amplitudes can be chosen nonnegative in this basis.

\subsection*{Bethe ansatz state in the computational basis}
Fix $M=L$ and write the Bethe state in the $\sigma^z$ basis as
\begin{equation}
  \ket{\Psi}
  \;=\;
  \sum_{1\le x_1<\cdots<x_L\le 2L}
  \psi(x_1,\dots,x_L)\;
  S^-_{x_1}\cdots S^-_{x_L}\ket{\!\uparrow^{\otimes 2L}\!}
  \;=\;
  \sum_{\sigma\in\{\uparrow,\downarrow\}^{2L}}
  \Psi(\sigma)\,\ket{\sigma},
\end{equation}
where $S_x^-=\tfrac12(\sigma_x^x-i\sigma_x^y)$ and $\Psi(\sigma)$ is nonzero only when exactly $L$
spins are down.  (In the coordinate Bethe ansatz, $\psi(x_1,\dots,x_L)$ is a plane--wave sum with
two--body phase shifts determined by the rapidities.)

\subsection*{Overlap formula $\psi(M)=\langle M|\Psi\rangle$}
Expanding each singlet into the $\sigma^z$ basis,
\begin{equation}
  \ket{s_{ab}}
  \;=\;
  \frac{1}{\sqrt{2}}
  \Big(\ket{\uparrow_a\downarrow_b}-\ket{\downarrow_a\uparrow_b}\Big),
\end{equation}
we obtain an \emph{exact} change--of--basis formula:
\begin{equation}
  \boxed{\quad
  \psi(M)\;=\;\braket{M}{\Psi}
  \;=\;2^{-L/2}\!
  \sum_{\sigma\;\text{s.t.}\; \sigma_a\neq\sigma_b\;\forall (a,b)\in M}
  \Bigg[
    \prod_{(a,b)\in M}\varepsilon(\sigma_a,\sigma_b)
  \Bigg]\,
  \Psi(\sigma)
  \quad}
  \label{eq:VBoverlap}
\end{equation}
with the \emph{dimer sign} function
\begin{equation}
  \varepsilon(\uparrow,\downarrow)=+1,\qquad
  \varepsilon(\downarrow,\uparrow)=-1,\qquad
  \varepsilon(\uparrow,\uparrow)=\varepsilon(\downarrow,\downarrow)=0.
\end{equation}
Equivalently, the selection rule ``$\sigma_a\neq\sigma_b$ for all $(a,b)\in M$''
says that only $\sigma$ with opposite spins on every matched pair contribute; the product of
$\varepsilon$'s provides the relative sign fixed by the $A\to B$ singlet orientation.

\paragraph{Remarks.}
\begin{itemize}
  \item Eq.~\eqref{eq:VBoverlap} is a constrained sum over $2^L$ configurations
  (not $2^{2L}$): once $\sigma_a$ is chosen on $A$, each matched $b$ is forced to be the opposite.
  \item If one writes $\Psi(\sigma)$ in the coordinate Bethe form,
  \(\psi(x_1,\dots,x_L)=\sum_{\pi\in S_L}A(\pi)e^{i\sum_j k_{\pi(j)}x_j}\),
  then \eqref{eq:VBoverlap} yields $\psi(M)$ as a structured linear combination of those plane waves,
  with coefficients dictated by the matching constraints.
  \item With the $A\to B$ orientation, $\psi(M)$ can be chosen real and (for the XXX antiferromagnet) nonnegative for the ground state.
\end{itemize}

\subsection*{Nonorthogonality and the dual VB coordinates}
Because $\{\ket{M}\}$ is overcomplete, the overlaps $\psi(M)=\langle M|\Psi\rangle$ are not
the unique coordinates of $\ket{\Psi}$.  Let $O_{MN}=\braket{M}{N}$ be the Gram matrix;
on a planar bipartite lattice (including the cycle) the \emph{loop rule} gives
\begin{equation}
  \braket{M}{N}
  \;=\; 2^{\ell(M,N)-L},
\end{equation}
where $\ell(M,N)$ is the number of disjoint loops in the transition graph formed by superimposing
$M$ and $N$.
Define the \emph{dual VB basis} $\{\ket{\widetilde M}\}$ by
\begin{equation}
  \ket{\widetilde M}\;=\;\sum_{N} \ket{N}\,(O^{-1})_{NM},
  \qquad\text{so that}\qquad
  \braket{\widetilde M}{N}=\delta_{MN}.
\end{equation}
Then the true coordinates of $\ket{\Psi}$ in this (biorthogonal) basis are
\begin{equation}
  c_M \;=\; \braket{\widetilde M}{\Psi}
  \;=\; \sum_{N} (O^{-1})_{MN}\,\braket{N}{\Psi}
  \;=\; \sum_{N} (O^{-1})_{MN}\,\psi(N).
\end{equation}
In particular, the singlet--sector identity resolves as
\begin{equation}
  \mathbb{I}_{S=0}
  \;=\; \sum_{M}\ket{M}\bra{\widetilde M}
  \;=\; \sum_{M,N}\ket{M}\,(O^{-1})_{MN}\bra{N}.
\end{equation}

\subsection*{Summary}
Given the Bethe state amplitudes $\Psi(\sigma)$ in the computational basis,
the change of basis to any VB state $\ket{M}$ is computed exactly by the constrained sum
\eqref{eq:VBoverlap}.  If one needs bona fide coordinates in a VB representation, use the
dual basis $\{\ket{\widetilde M}\}$ and contract with $O^{-1}$ as above.


\section*{Quantum Max-Cut on the even cycle and the two-dimer ansatz}

\paragraph{Setup.}
Let $C_{2L}$ be the cycle with sites $1,\dots,2L$ and periodic boundary $(2L{+}1)\equiv 1$.
We work with spin-$\tfrac12$ operators $S_i^\alpha=\tfrac12 \sigma_i^\alpha$ so that
\[
\mathbf S_i\!\cdot\!\mathbf S_j=\frac14\,\boldsymbol\sigma_i\!\cdot\!\boldsymbol\sigma_j.
\]
Denote by $P_-^{(ij)}$ the projector onto the singlet on edge $(i,j)$:
\[
P_-^{(ij)} \;=\; \frac14\bigl(\mathbf 1 - \boldsymbol\sigma_i\!\cdot\!\boldsymbol\sigma_j\bigr)
\;=\; \frac14\,\mathbf 1 \;-\; \mathbf S_i\!\cdot\!\mathbf S_j.
\]

\paragraph{Hamiltonians and scaling.}
The (nearest-neighbor) \emph{quantum Max-Cut} Hamiltonian on $C_{2L}$ is
\[
H_{\mathrm{QMC}} \;=\; \sum_{j=1}^{2L} P_-^{(j,j+1)}.
\]
With the above identity,
\[
H_{\mathrm{QMC}}
\;=\; \sum_{j=1}^{2L}\!\Bigl(\frac14\mathbf 1 - \mathbf S_j\!\cdot\!\mathbf S_{j+1}\Bigr)
\;=\; \frac{2L}{4}\,\mathbf 1 \;-\; H_{\mathrm{Heis}},
\qquad
H_{\mathrm{Heis}}:=\sum_{j=1}^{2L}\mathbf S_j\!\cdot\!\mathbf S_{j+1}.
\]
Thus the two models are \emph{affinely equivalent}:
they have the \emph{same eigenstates}, and their spectra are related by
\[
E_{\mathrm{QMC}} \;=\; \frac{L}{2} - E_{\mathrm{Heis}}.
\]
All finite-$L$ statements for one transfer to the other via this shift.

\paragraph{Two perfect matchings and the RVB doublet.}
Let $M_A$ and $M_B$ be the two nearest-neighbor perfect matchings (dimers) on $C_{2L}$:
\[
M_A=\{(1,2),(3,4),\ldots,(2L-1,2L)\},\qquad
M_B=\{(2,3),(4,5),\ldots,(2L,1)\}.
\]
Write $|M_A\rangle$ and $|M_B\rangle$ for the corresponding product-of-singlets VB states.
Their overlap is
\[
s:=\langle M_A|M_B\rangle \;=\; 2^{1-L}.
\]
Define the \emph{normalized} symmetric/antisymmetric superpositions
\[
|\psi_\pm\rangle \;=\; \frac{|M_A\rangle \pm |M_B\rangle}{\sqrt{2(1\pm s)}}.
\]

\paragraph{Matrix elements you need once.}
(i) On a single covering,
\[
\langle M_A|\,\mathbf S_j\!\cdot\!\mathbf S_{j+1}\,|M_A\rangle
=\begin{cases}
-\tfrac{3}{4},& (j,j{+}1)\in M_A,\\[2pt]
0,& \text{otherwise,}
\end{cases}
\qquad\Rightarrow\qquad
\langle M_A|H_{\mathrm{Heis}}|M_A\rangle \;=\; -\frac{3}{4}L.
\]
(ii) For the \emph{cross} terms on the even cycle (neighbors lie on opposite sublattices),
\[
\langle M_A|\,\mathbf S_j\!\cdot\!\mathbf S_{j+1}\,|M_B\rangle \;=\; -\frac{3}{4}\,s
\quad\text{for every }j,
\qquad\Rightarrow\qquad
\langle M_A|H_{\mathrm{Heis}}|M_B\rangle \;=\; -\frac{3}{4}(2L)\,s.
\]
(iii) Using $P_-^{(j,j+1)}=\tfrac14\mathbf 1-\mathbf S_j\!\cdot\!\mathbf S_{j+1}$,
\[
\langle M_A|H_{\mathrm{QMC}}|M_A\rangle
= \sum_{j}\!\bigl\langle M_A|P_-^{(j,j+1)}|M_A\bigr\rangle
= L\cdot 1 \;+\; L\cdot \frac14 \;=\; \frac{5}{4}L,
\]
and
\[
\langle M_A|H_{\mathrm{QMC}}|M_B\rangle
= \sum_{j}\!\Bigl(\frac14 s - \langle M_A|\mathbf S_j\!\cdot\!\mathbf S_{j+1}|M_B\rangle\Bigr)
= \sum_{j}\!\Bigl(\frac14 s + \frac{3}{4}s\Bigr)
= (2L)\,s.
\]
(All these follow immediately from the Heisenberg expressions via the affine relation.)

\paragraph{Variational energy at finite $L$.}
For the symmetric state $|\psi_+\rangle$ one obtains
\[
E_{\mathrm{Heis}}^{(+)}(L)
=\frac{\langle \psi_+|H_{\mathrm{Heis}}|\psi_+\rangle}{1}
= \frac{-\tfrac{3}{4}L \;-\; \tfrac{3}{4}(2L)s}{1+s}
= -\frac{3L}{4}\,\frac{1+2s}{1+s},
\qquad s=2^{1-L}.
\]
Consequently, for the quantum Max-Cut Hamiltonian,
\[
E_{\mathrm{QMC}}^{(+)}(L)
= \frac{L}{2} - E_{\mathrm{Heis}}^{(+)}(L)
= \frac{L}{2} + \frac{3L}{4}\,\frac{1+2s}{1+s}
= L\;\frac{5+2s}{4(1+s)}.
\]
The corresponding \emph{energy per edge} $e^{(+)}_{\mathrm{QMC}}(L):=E^{(+)}/(2L)$ and
\emph{per site} $E^{(+)}/(2L)$ are
\[
e^{(+)}_{\mathrm{QMC}}(L)=\frac{1}{2}\,\frac{5+2s}{4(1+s)},
\qquad
\varepsilon^{(+)}_{\mathrm{QMC}}(L):=\frac{E^{(+)}_{\mathrm{QMC}}}{2L}
= \frac{5+2s}{8(1+s)}.
\]
Exactly the same algebra gives the antisymmetric value with $s\mapsto -s$.

\paragraph{Checks at small sizes.}
For $L=2$ ($N=4$) one has $s=\tfrac12$ and
\[
E_{\mathrm{Heis}}^{(+)}(2)=-2,\qquad
E_{\mathrm{QMC}}^{(+)}(2)= \frac{L}{2}-E_{\mathrm{Heis}}^{(+)}(2)=1+2=3,
\]
which matches the fact that $|M_A\rangle\pm|M_B\rangle$ span the exact ground-state
doublet (Heisenberg) and, by the affine relation, the exact top/bottom of $H_{\mathrm{QMC}}$
depending on convention.

\paragraph{Finite-$L$ exactness via Bethe ansatz.}
Because $H_{\mathrm{QMC}}$ differs from $H_{\mathrm{Heis}}$ by a multiple of the identity, the
\emph{eigenstates coincide} at every finite $L$. Thus the Bethe ansatz furnishes an
\emph{exact finite-$L$ solution} for $H_{\mathrm{QMC}}$:
solve the XXX Bethe equations for $N=2L$ and $M=N/2$ down-spins,
\[
\left(\frac{\lambda_j+\tfrac{i}{2}}{\lambda_j-\tfrac{i}{2}}\right)^{\!N}
= \prod_{k\ne j}^{M}\frac{\lambda_j-\lambda_k+i}{\lambda_j-\lambda_k-i},
\qquad j=1,\dots,M,
\]
and obtain the Heisenberg energies
\[
E_{\mathrm{Heis}} = -\sum_{j=1}^{M}\frac{1}{\lambda_j^2+\tfrac14},
\]
in the above normalization. The corresponding quantum Max-Cut energies at finite $L$ are then
\[
E_{\mathrm{QMC}} \;=\; \frac{L}{2} - E_{\mathrm{Heis}}.
\]
No thermodynamic ($L\to\infty$) limit is required: all quantities are exact for any fixed $L$.

\paragraph{Conclusion.}
With the “two-dimer” variational state $|\psi_\pm\rangle$, the \emph{finite-$L$} expectation
of $H_{\mathrm{QMC}}$ is
\[
E_{\mathrm{QMC}}^{(\pm)}(L)=L\,\frac{5\pm 2s}{4(1\pm s)},\qquad s=2^{1-L}.
\]
The \emph{exact} finite-$L$ ground-state energy (and gap, etc.) follow from the Bethe equations
for the XXX chain and the affine shift $E_{\mathrm{QMC}}=L/2-E_{\mathrm{Heis}}$.




\end{document}
