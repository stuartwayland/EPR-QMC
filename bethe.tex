% % Max-Cut Hamiltonian notes
% \documentclass[11pt]{article}
% \usepackage[margin=1in]{geometry}
% % (removed duplicate \documentclass[12pt]{article})

% % === Core math packages ===
% \usepackage{amsmath, amssymb, amsthm, bm}

% % === Optional but common ===
% \usepackage{physics}    % for \dv, \ket{}, \bra{}, etc. (not strictly needed here)
% \usepackage{braket}     % lightweight alternative if you only need bras/kets

% % === Fonts and symbols ===
% \usepackage{bbm}        % for identity operator \mathbbm{1} if you prefer it to I
% \usepackage{mathtools}  % math enhancements (e.g., \coloneqq)

% % === Optional for small spacing improvements ===
% \usepackage{microtype}



% \begin{document}
% \section*{Reduction of QMC Hamiltonian to the Heisenberg XXX Chain}

% Let $C_L$ denote the cycle graph on $L$ vertices with periodic boundary
% conditions $(i+L \equiv i)$.
% For each edge $(i,i+1)$ define the two-qubit operator
% \[
% H_{i,i+1}
% \;=\;
% \frac{1}{4}\!\left(
% I
% - X_i X_{i+1}
% - Y_i Y_{i+1}
% - Z_i Z_{i+1}
% \right),
% \]
% so that the total QMC Hamiltonian is
% \[
% H_{\mathrm{QMC}}
% \;=\;
% \sum_{i=1}^{L} H_{i,i+1}
% \;=\;
% \frac{1}{4}\sum_{i=1}^{L}
% \!\left(
% I
% - X_i X_{i+1}
% - Y_i Y_{i+1}
% - Z_i Z_{i+1}
% \right).
% \]

% Introduce the spin operators
% \[
% S_i^{x}=\tfrac{1}{2}X_i,\qquad
% S_i^{y}=\tfrac{1}{2}Y_i,\qquad
% S_i^{z}=\tfrac{1}{2}Z_i,
% \qquad
% \bm S_i=(S_i^{x},S_i^{y},S_i^{z}).
% \]
% Then
% \[
% \bm S_i\!\cdot\!\bm S_{i+1}
% =
% S_i^{x}S_{i+1}^{x}
% +S_i^{y}S_{i+1}^{y}
% +S_i^{z}S_{i+1}^{z}
% =\frac{1}{4}\!\left(
% X_i X_{i+1}
% +Y_i Y_{i+1}
% +Z_i Z_{i+1}
% \right).
% \]

% Substituting into $H_{\mathrm{QMC}}$ gives
% \[
% H_{\mathrm{QMC}}
% =
% \frac{1}{4}\sum_{i=1}^{L} I
% \;-\;
% \sum_{i=1}^{L} \bm S_i\!\cdot\!\bm S_{i+1}
% =
% \frac{L}{4}I
% \;-\;
% H_{\mathrm{XXX}},
% \]
% where
% \[
% H_{\mathrm{XXX}}
% \;=\;
% \sum_{i=1}^{L} \bm S_i\!\cdot\!\bm S_{i+1}
% \]
% is the spin-$\tfrac12$ isotropic Heisenberg (XXX) Hamiltonian
% with periodic boundary conditions.

% Hence
% \[
% E_{\mathrm{QMC}}^{\max}(L)
% \;=\;
% \frac{L}{4}
% -
% E_{\mathrm{XXX}}^{\min}(L),
% \]
% so maximizing the QMC objective is equivalent to finding
% the antiferromagnetic ground state of the XXX chain.

% \section*{Bridge: Bethe Ansatz and Matchings on the Cycle Graph}

% Consider the QMC Hamiltonian on the even cycle $C_L$,
% \[
% H_{\mathrm{QMC}}
% =\sum_{i=1}^{L} e_i,
% \qquad
% e_i
% =\frac{1}{4}\!\left(I
% - X_iX_{i+1}
% - Y_iY_{i+1}
% - Z_iZ_{i+1}\right),
% \]
% with periodic $i+L\equiv i$.
% Each $e_i$ is the projector onto the singlet subspace
% of qubits $(i,i+1)$:
% \[
% e_i = \dyad{\psi^-_{i,i+1}},
% \qquad
% \ket{\psi^-_{a,b}}
% =\tfrac{1}{\sqrt{2}}\!\left(\ket{01}-\ket{10}\right)_{ab}.
% \]

% \subsection*{Valence-Bond (VB) Basis and Matchings}

% A \emph{perfect matching} $M$ of $\{1,\dots,L\}$ is a partition into
% $L/2$ disjoint pairs $(a,b)$.
% Each matching defines a valence-bond state
% \[
% \ket{\mathrm{VB}(M)}
% =\bigotimes_{(a,b)\in M}
% \ket{\psi^-_{a,b}}.
% \]
% The collection of all such $\ket{\mathrm{VB}(M)}$ spans
% the total-spin singlet subspace of $(\mathbb{C}^2)^{\!\otimes L}$.

% \subsection*{Temperley--Lieb Action}

% The operators $\{e_i\}$ satisfy the
% Temperley--Lieb relations
% \[
% e_i^2 = e_i,\qquad
% e_i e_{i\pm1} e_i = e_i,\qquad
% [e_i,e_j]=0\quad (|i-j|>1).
% \]
% Their action on the VB basis is purely combinatorial:
% \begin{align*}
% e_i\ket{\mathrm{VB}(M)}
% &=
% \begin{cases}
% \ket{\mathrm{VB}(M)}, & (i,i{+}1)\in M,\\[3pt]
% \ket{\mathrm{VB}(M')}, & \text{otherwise,}
% \end{cases}
% \end{align*}
% where $M'$ is obtained from $M$ by reconnecting the two bonds
% that meet sites $i$ and $i{+}1$.
% Thus $H_{\mathrm{QMC}}$ acts as a
% local ``reconnection'' operator on matchings.

% \subsection*{RVB Superposition and Bethe Ground State}

% Because $\{e_i\}$ mix matchings,
% no single $\ket{\mathrm{VB}(M)}$ is an eigenstate.
% The true ground state of
% $H_{\mathrm{QMC}}$ (equivalently of the XXX chain)
% is a resonating valence-bond (RVB) superposition
% \[
% \ket{\Psi_0}
% =\sum_{M} \psi(M)\,\ket{\mathrm{VB}(M)},
% \]
% with amplitudes $\psi(M)$ determined implicitly by
% the Bethe-ansatz wavefunction of the XXX model.
% Expectation values in $\ket{\Psi_0}$ correspond to probabilities
% over matchings:
% \[
% \ev{e_i}{\Psi_0}
% =
% \sum_{M}
% |\psi(M)|^2\,\mathbf{1}\!\bigl[(i,i{+}1)\in M\bigr].
% \]

% \subsection*{Edge Occupancy from Bethe Ansatz}

% From the Bethe ground-state energy density
% $e_0 = \tfrac{1}{4} - \ln 2$ for the XXX chain,
% and the mapping
% $H_{\mathrm{QMC}}=\tfrac{L}{4}I-H_{\mathrm{XXX}}$,
% one finds
% \[
% \frac{1}{L}\sum_{i=1}^{L}\ev{e_i}{\Psi_0}
% =\ln 2.
% \]
% Hence the Bethe ansatz determines the stationary
% distribution over perfect matchings in which, on average,
% each edge of the cycle is occupied by a singlet
% with probability $\ln 2$.
% This provides a direct combinatorial interpretation
% of the Bethe ground state as an optimal
% superposition over all matchings of $C_L$.


% \end{document}
\documentclass[11pt]{article}
\usepackage{amsmath,amssymb,amsthm,bm}
\usepackage[margin=1in]{geometry}

\newcommand{\ket}[1]{\left|#1\right\rangle}
\newcommand{\bra}[1]{\left\langle#1\right|}
\newcommand{\braket}[2]{\left\langle #1 \middle| #2 \right\rangle}

\begin{document}

\begin{center}
\Large QMC on Even Cycles
\end{center}

\section*{1. QMC Hamiltonian on an even cycle and mapping to XXX}
Let $C_{2L}$ be the even cycle with sites $i=1,\dots,2L$ and periodic boundary $i\equiv i+2L$.
The (sign-problem-free) ``QMC'' Hamiltonian we use is the antiferromagnetic Heisenberg (XXX) model
\begin{equation}
  H_{\mathrm{XXX}}
  \;=\;
  J\sum_{i=1}^{2L} \bm S_i\cdot \bm S_{i+1},
  \qquad J>0,\quad \bm S_i=\tfrac12(\sigma_i^x,\sigma_i^y,\sigma_i^z).
  \label{eq:XXX}
\end{equation}
This is unfrustrated on $C_{2L}$ (bipartite, $|A|=|B|=L$) and SU(2)-invariant. Total spin $S_{\rm tot}$ is a good quantum number; by Lieb--Mattis on balanced bipartite graphs, the ground state is a singlet: $S_{\rm tot}=0$.

\smallskip
\noindent
\emph{Remark on QMC/EPR variants.}
If one starts from a ``quantum Max-Cut'' form with $\sigma^x_i\sigma^x_{i+1}$ couplings, a global basis rotation around the $y$ axis maps it to \eqref{eq:XXX} up to constants on the even cycle. Hence we take \eqref{eq:XXX} as the canonical form.

\section*{2. Bethe ansatz solution (periodic, even length, $S^z_{\rm tot}=0$)}
Work in the computational basis with $M=L$ flipped spins:
\begin{equation}
  \ket{\Psi}
  = \sum_{1\le x_1<\cdots<x_M\le 2L}
  \psi(x_1,\dots,x_M)\, S^-_{x_1}\cdots S^-_{x_M}\ket{\!\uparrow^{\otimes 2L}\!}.
\end{equation}
The \emph{coordinate Bethe ansatz} amplitude is
\begin{equation}
  \psi(x_1,\dots,x_M)=
  \sum_{P\in S_M} A(P)\,
  \exp\!\Big(i\sum_{j=1}^{M} k_{P_j}\, x_j\Big),
  \qquad
  A(P)=\prod_{1\le j<\ell\le M}
  \frac{\sin\frac{k_{P_j}-k_{P_\ell}}{2}-i}{\sin\frac{k_{P_j}-k_{P_\ell}}{2}+i},
\end{equation}
with quasi-momenta $\{k_j\}_{j=1}^{M}$ solving the Bethe equations
\begin{equation}
  e^{i k_j (2L)}=\prod_{\ell\neq j}
  \frac{\sin\frac{k_j-k_\ell}{2}+i}{\sin\frac{k_j-k_\ell}{2}-i},
  \qquad j=1,\dots,M.
\end{equation}
For the ground state, the $k_j$ are real and symmetrically fill a Fermi sea. The ground-state energy is
\begin{equation}
  E_0
  = \sum_{j=1}^{M}\,\big( \tfrac{J}{2}(\cos k_j -1)\big),
\end{equation}
and the state is a unique translationally invariant singlet ($S_{\rm tot}=0$).

\section*{3. Valence-bond (VB) basis for the singlet sector}
Let $\mathcal{M}$ be the set of perfect matchings $M$ on $\{1,\dots,2L\}$. For $(i,j)\in M$ define the singlet on sites $i,j$:
\(
\ket{\psi^-_{ij}}=\tfrac{1}{\sqrt{2}}\big(\ket{\uparrow_i\downarrow_j}-\ket{\downarrow_i\uparrow_j}\big).
\)
A VB basis vector is the tensor product
\begin{equation}
  \ket{\mathrm{VB}(M)} \;=\; \bigotimes_{(i,j)\in M}\ket{\psi^-_{ij}}.
\end{equation}
The set $\{\ket{\mathrm{VB}(M)}: M\in\mathcal{M}\}$ spans the full $S_{\rm tot}=0$ subspace (overcomplete). Since the Bethe ground state is a singlet, it admits an exact expansion
\begin{equation}
  \ket{\Psi_0}
  \;=\;
  \sum_{M\in\mathcal{M}} \psi(M)\,\ket{\mathrm{VB}(M)}.
  \label{eq:VBexpansion}
\end{equation}

\section*{4. Distance on the even cycle and the amplitude-product ansatz}
On the ring, define the \emph{cycle distance} between sites $i,j$:
\begin{equation}
  r_{ij} \;=\; \min\{|i-j|,\;2L-|i-j|\}\in\{1,\dots,L\}.
\end{equation}
Empirically and from field-theory/Bethe correlators, the single-bond length distribution in the ground state obeys
\begin{equation}
  P(r)\;\equiv\; \frac{1}{2L}\,\sum_{i=1}^{2L}\sum_{j:\, r_{ij}=r}
  \big\langle \Psi_0 \big| \hat n_{ij} \big|\Psi_0\big\rangle
  \;\sim\; \frac{c}{r^2}\qquad (1\ll r\ll L),
\end{equation}
where $\hat n_{ij}$ projects onto the event that $i$ and $j$ are paired in a VB configuration within the (overlap) estimator.
This motivates the \emph{amplitude-product RVB} form for the VB coefficients in \eqref{eq:VBexpansion}:
\begin{equation}
  \boxed{\quad
  \psi(M) \;\propto\; \prod_{(i,j)\in M} h(r_{ij}),
  \qquad
  h(r)\;\approx\; \frac{(-1)^r}{r^{\alpha}},\ \ \alpha\simeq 1,
  \quad}
  \label{eq:APRVB}
\end{equation}
so that the Born weights $|\psi(M)|^2$ produce $P(r)\sim r^{-2}$ asymptotically. The staggered sign $(-1)^r$ enforces bipartite Marshall signs (sites alternate along the ring).

\paragraph{Normalization and calibration.}
Let $Z=\sum_{M}\prod_{(i,j)\in M}|h(r_{ij})|^2$. Then the induced bond-length law satisfies
\begin{equation}
  P(r)\;=\;\frac{1}{2L}\sum_{i=1}^{2L}\sum_{j:\,r_{ij}=r}\;
  \frac{1}{Z}\sum_{M\ni (i,j)} \prod_{(a,b)\in M}|h(r_{ab})|^2,
  \qquad \sum_{r=1}^{L}P(r)=1.
\end{equation}
Choosing $h(r)\propto r^{-1}$ matches the Bethe prediction $P(r)\sim r^{-2}$ at large $r$; short-distance corrections can be absorbed by a cutoff or a prefactor $h(r)=r^{-1}e^{-r/\xi}$ with $\xi\gg 1$.

\section*{5. Summary (chain $\to$ Bethe $\to$ VB amplitudes)}
\begin{itemize}
  \item On the even cycle, the antiferromagnetic XXX Hamiltonian \eqref{eq:XXX} is SU(2)-invariant and unfrustrated; the ground state is a unique singlet.
  \item The Bethe ansatz yields the exact ground state in the $M=L$ sector and predicts algebraic spin correlations; in VB language this corresponds to an algebraic tail $P(r)\sim r^{-2}$ of bond lengths.
  \item Expanding the Bethe singlet in the VB basis \eqref{eq:VBexpansion} is well approximated (and asymptotically matched) by the amplitude-product form \eqref{eq:APRVB} with $h(r)\propto r^{-1}$ (up to short-distance normalization and Marshall signs).
\end{itemize}

\end{document}
