% Max-Cut Hamiltonian notes
\documentclass[11pt]{article}
\usepackage[margin=1in]{geometry}
\usepackage{amsmath,amssymb,amsfonts}
\usepackage{braket}

% Dirac notation shortcuts
\newcommand{\ip}[2]{\langle #1 \mid #2 \rangle}


\begin{document}
\section*{Reduction of QMC Hamiltonian to the Heisenberg XXX Chain}

Let $C_L$ denote the cycle graph on $L$ vertices with periodic boundary
conditions $(i+L \equiv i)$.
For each edge $(i,i+1)$ define the two-qubit operator
\[
H_{i,i+1}
\;=\;
\frac{1}{4}\!\left(
I
- X_i X_{i+1}
- Y_i Y_{i+1}
- Z_i Z_{i+1}
\right),
\]
so that the total QMC Hamiltonian is
\[
H_{\mathrm{QMC}}
\;=\;
\sum_{i=1}^{L} H_{i,i+1}
\;=\;
\frac{1}{4}\sum_{i=1}^{L}
\!\left(
I
- X_i X_{i+1}
- Y_i Y_{i+1}
- Z_i Z_{i+1}
\right).
\]

Introduce the spin operators
\[
S_i^{x}=\tfrac{1}{2}X_i,\qquad
S_i^{y}=\tfrac{1}{2}Y_i,\qquad
S_i^{z}=\tfrac{1}{2}Z_i,
\qquad
\bm S_i=(S_i^{x},S_i^{y},S_i^{z}).
\]
Then
\[
\bm S_i\!\cdot\!\bm S_{i+1}
=
S_i^{x}S_{i+1}^{x}
+S_i^{y}S_{i+1}^{y}
+S_i^{z}S_{i+1}^{z}
=\frac{1}{4}\!\left(
X_i X_{i+1}
+Y_i Y_{i+1}
+Z_i Z_{i+1}
\right).
\]

Substituting into $H_{\mathrm{QMC}}$ gives
\[
H_{\mathrm{QMC}}
=
\frac{1}{4}\sum_{i=1}^{L} I
\;-\;
\sum_{i=1}^{L} \bm S_i\!\cdot\!\bm S_{i+1}
=
\frac{L}{4}I
\;-\;
H_{\mathrm{XXX}},
\]
where
\[
H_{\mathrm{XXX}}
\;=\;
\sum_{i=1}^{L} \bm S_i\!\cdot\!\bm S_{i+1}
\]
is the spin-$\tfrac12$ isotropic Heisenberg (XXX) Hamiltonian
with periodic boundary conditions.

Hence
\[
E_{\mathrm{QMC}}^{\max}(L)
\;=\;
\frac{L}{4}
-
E_{\mathrm{XXX}}^{\min}(L),
\]
so maximizing the QMC objective is equivalent to finding
the antiferromagnetic ground state of the XXX chain.
\end{document}
